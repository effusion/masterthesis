\graphicspath{{./images/}}

\chapter{Ausgangslage}

\section{Ist-Zustand}

Die Applikation MEON ist seit April 2016 in Betrieb und wird aktuell gewartet und weiterentwickelt. Die Applikation wurde unter speziellen Bedienungen im Zuge der beiden DevOps Initiativen HAKA6 und SuperSonic\footnote{Firmeninterne Programme bei denen es um die Einführung von DevOps Praktiken geht. DevOps hat das Ziel Entwicklung und Betrieben näher zusammen zubringen.} umgesetzt, so dass gewisse Technologien mit Unterstützung dieser Initiativen benutzt werden durften, die zu diesem Zeitpunkt noch nicht produktiv betrieben wurden. Der Marktdruck und diese Initiativen ermöglichten eine enge Zusammenarbeit zwischen Business und Entwicklung, bei welcher die Anforderungen gemeinsam erarbeitet und ohne formale Dokumente direkt vom Product Owner ins Team gegeben wurden, weshalb keine entsprechenden Requirements Dokumente vorliegen.
\newpage
\subsection{Technisch}

Die Applikation wurde mit aktuellen Frameworks entwickelt. In der folgenden Tabelle sind die wichtigsten Komponenten und ihre Funktionen kurz aufgelistet.

\begin{table}[H]
	\centering
	\caption{Technische Rahmenbedienungen}
	\begin{tabular}{ | p{4cm} | p{11cm} | }
		\toprule
		{\textbf{Technologie / Tool}} & {\textbf{Einsatzzweck}} \\
		\midrule
		Java 8 & Programmiersprache mit der die Serverapplikation implementiert ist. \\ \hline
		Java Script & Programmiersprache mit der die Webapplikation implementiert ist. \\ \hline
		Spring Framework & Basisframework für die Backend-Applikation.  \\ \hline
		AngularJS & Webframework für die Webapplikation. \\ \hline
		Docker & Container Software für das Deployment der einzelnen Teile der Applikation. \\ \hline
		Wildfly 10 & Applikationsserver welcher die Laufzeitumgebung für die Applikation bereitstellt. \\ \hline
		Apache Webserver & HTTP Server für den statischen Inhalt. \\ \hline
		Camunda BPM & Workflow-Management-System welches die Prozesse für das Onboarding eines neuen Händlers steuert. \\ \hline
		Red Hat Enterprise Linux & Basis Betriebssystem für Server und Container.\\ \hline
		Oracle, MongoDB & Datenspeicher. \\ \hline
		SIX Build Umgebung & Umgebung welche Tools für Source Code Verwaltung(GIT), Build Server(Jenkins) und Qualität(Sonar) zur Verfügung stellt.\\ \hline
		Astah Professional & Tool für das Zeichen von Diagrammen.\\ \hline
		Gantt Project & Tool für das Erstellen von Projektplänen.\\		
		\bottomrule
	\end{tabular}
\end{table}
\noindent Die Anwendung ist technologisch auf einem sehr modernen Stand. Der Einsatz der Containertechnologie Docker hat sich bewährt und wird auch in anderen Projekten eingesetzt. Ob der weitere Einsatz von Wildfly, respektive eines Applikationsservers, notwendig ist, kann in Frage gestellt werden. Dies vor allem durch die teilweise Unklarheit der Stossrichtung der Java Enterprise Edition. Obschon Oracle bereits Pläne vorgelegt hat, sind nicht alle Zweifel über die Zukunft ausgeräumt. Da die Applikation bereits auf Spring basiert könnte hier der Schritt Richtung Spring Boot\footnote{Basiert auf dem Spring Framework benötigt jedoch keinen Applikationserver mehr da dieser enthalten ist.\url{https://projects.spring.io/spring-boot/}} mit verhältnismässig geringem Aufwand durchgeführt werden. Im folgenden Bild ist die aktuelle Verteilungssicht der Applikation dargestellt.
\begin{figure}[H]
	\centering
	\includegraphics[scale=0.6]{CurrentDeployment.png}
	\caption{Verteilungssicht}
\end{figure}
\newpage \noindent 
Der Onboarding Teil ist mit Hilfe einer Schichtenarchitektur umgesetzt wobei die Präsentationsschicht auf dem Webserver liegt, die Business, Persistent- und Integrationsschicht auf dem Applikationsserver. Die Workflow Engine ist ein Produkt der Firma Camunda und ist als Anwendungsarchiv vorhanden welches zusammen mit dem WildFly Server ausgerollt wird. MEON wird bei jedem Commit in das Source Code Repository gebaut und auch Testsystemen Installiert.\newline
Die Applikation läuft in beiden Rechenzentren gleichzeitig wodurch sie eine hohe Ausfallsicherheit bietet. Die Installation wird halbautomatisch durchgeführt und hat zu Folge, dass für eine kurze Zeit zwei Applikationsversionen in Betrieb sind. Diese sind jedoch nicht kompatible wodurch es zu Probleme bei der Datenspeicherung oder Datenübertragung kommen könnte.

\subsection{Organisatorisch}

Da SIX im Finanzbereich tätig ist, gibt es Richtlinien und Prozesse wann, wie und ob Änderungen in Produktion gehen dürfen. Aktuell werden die Changes im Issue Trackingsystem Jira erfasst und mit einer Release Version versehen. Durch die Anbindung an Fisheye und SVN ist jederzeit ersichtlich mit welchem Ticket welche Codeteile betroffen waren. Da das Projekt unter erleichterten Bedingungen umgesetzt wurde, ist ein Eintrag im ITSM \footnote{Anwendung für ITIL\url{https://de.wikipedia.org/wiki/IT_Infrastructure_Library} welches auch das Change Management enthält.} System aktuell nicht Pflicht obschon das gemäss normalen Richtlinien der Fall wäre. Mittlerweile können Änderungen mit einem Prozess von Jira ins ITSM automatisch übertragen werden.
Gemäss den Richtlinien müsste eine Änderung fünf Personentage vor Going live eingetragen und freigegeben sein. Die Praxis ist wie erwähnt auf die Branche der SIX zurückzuführen und hat unter genauerer Betrachtung eher formellen Nutzen. Die Freigabe im System wird meistens von Personen gemacht welche nicht an der Entwicklung und dem Betrieb der Applikation beteiligt sind. Für diese Personen ist eine Beurteilung der Änderungen meist nicht möglich. Hier wäre wohl eine Kategorisierung der Applikation, welche einen solchen Prozess benötigen, sinnvoll, um nicht unnötige Prozesse für kleine nicht zwingen kritische Applikation zu forcieren.

\section{Situationsanalyse}

SIX befindet sich aktuell im Bezug auf Betrieb und Entwicklung von Applikationen im Wandel. Die bereits erwähnten DevOps Initiativen sind im Gange und werden von immer mehr Projekten angewendet, ja sogar gelebt. In diesem Zug ist aktuell auch eine OpenShift 3 \footnote{Software welche eine Cloud Umgebung für Container bereitstellt. Als Container Technologie wird Docker verwendet. Für die Orchestration der Container, wird Kubernetes von Google verwendet.} Installation im Gange, welche die Bereitstellung neuer Applikationen mit einem Cloud Ansatz noch weiter vereinfachen soll.
Durch die Initiativen ändert sich auch der organisatorische Aspekt. So sollen Value Streams erschaffen werden, in welchen eine End-To-End Verantwortung für Applikation wahrgenommen wird und Entscheide dezentral gefällt werden können. Insbesondere das Change Management muss sich zur Zeit an die neuen Methoden gewöhnen und damit umgehen lernen.

\section{Methoden}

\subsection{Scrum}

Für das Projekt Management der Masterthesis wird Scrum\footnote{Iteratives Entwicklungsverfahren welches sich auf das Agile Manifesto stützt und diesem diverse Rollen hinzufügt.} verwendet. Im Rahmen der Masterthesis wird jedoch nicht der ganze Werkzeugkasten der Methodik verwendet da dies zu viel Aufwand bedeuten würde. Wichtige Teile, wie das Planning, werden entsprechen kurz gehalten.
Scrum eignet sich gut für die Umsetzung der Arbeit da es implizit einen kurzen Feedbackzyklus vorschreibt. Da es sich beim Thema für SIX um Neuland handelt und daher keine Erfahrungen vorliegen, sind kleine iterative Schritte gut um Risiken früh zu erkennen und Anpassungen schnell umzusetzen.

\subsection{Prototyping}
Die neu entstehende Architektur soll auf ihre Tauglichkeit verifiziert werden. Da die Applikation bereits im Betrieb ist und aktuell stark erweitert wird, soll die Umsetzung mittels Prototyping geprüft werden. Hierbei werden die aus Architektursicht kritischen Bestandteile und deren Technologien und Prinzipien getestet. Für das automatische Testing stehen bereits Mockimplemetationen zu Verfügung, welche ebenfalls in den Prototypen verwendet werden können. Dadurch kann schnell eine Aussage über die Verwendbarkeit getroffen werden. 
