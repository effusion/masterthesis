\graphicspath{{./images/}}

\chapter{Ausgangslage}

\section{Ist-Zustand}

Die Applikation ist aktuell seit xxx in Betrieb und wird regelmässig erweitert. 

Die Applikation MEON wird aktuell mittels Continuous Delivery \footnote{Methode welches es der Entwicklung erlaubt Applikationen schnell auf nicht produktiven System auszurollen.} auf interne Testsysteme installiert. 

\section{Situationsanalyse}

Für die Umsetzung der Masterthesis wird Scrum verwendet. Scrum ist eine Agile Vorgehensmethodik welche nur wenige Regeln vorgibt. Die Wahl fiel auf diese Methodik da sie erstens sehr schlank ist und schnelle Feedbackzyklen erlaubt wodurch die Lösung besser auf die Bedürfnisse der Stakeholder abgestimmt werden kann. Des Weiteren können Probleme und Risiken dadurch früher entdeckt werde. Scrum ist in der Abteilung, in welcher die Masterthesis gemacht wird, die Standard Methodik für Software Projekte. Dabei werden nicht nur Entwicklungs- sondern auch, Analyse-, Anforderungs- und Testaufgaben damit abgehandelt. Die Position des Product Owners wird von meinem Betreuer eingenommen. Die Sprintdauer wird auf zwei Wochen angesetzt. Da die Masterthesis eine Einzelarbeit ist, werden andere Scrum Prozesse wie Retrospektive und Planning entsprechen vereinfacht oder ganz weggelassen. Scrum bietet den Rahmen für das Erstellen eines Software Architektur Dokumentes auf der Basis von Arc42.
Prototyping soll verwendet werden um die ausgewählten Lösungsstrategien gegenüber den Anforderungen zu verifizieren und gegebenenfalls anzupassen.  

Im vergleich zu anderen Projekte sehr modern