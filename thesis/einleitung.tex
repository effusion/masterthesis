\chapter{Einleitung}

\section{Problemstellung}

Im Zuge der Digitalisierung ist es für Finanzdienstleister unerlässlich die angebotenen Dienstleistungen schneller weiterzuentwickeln und der Kundschaft bereitzustellen. Vor Allem bei Online-Diensten wird diese immer wichtiger. Mit der Entwicklung von Paymit war eine Lösung notwendig welche es potenziellen neuen Händler einfach erlaubt sich für den neuen Dienst anzumelden. Aus dieser Anforderung ist die Applikation Merchant Onboarding entstanden welches einem neuen Kunde erlaubt sich einfach über eine Web-Oberfläche zu registrieren.

Die Applikation wurde basierend auf neueren Technologien und Methoden entwickelt, welche es erlauben die Applikation schneller von der Entwicklung in die Produktion zu bringen. Im Vergleich zu normalen Releasezyklen von drei Monaten ist diese ein grosser Schritt forwärt.

Obschon die Applikation mittlerweile innert Tagesfrist ausgerollt werden kann, soll mit der Anpassung der Architektur auf Continuous Deployment \footnote{Methode welche es der Entwicklung und dem Betrieb erlaubt, eine Applikation automatisch nach jedem Code Commit neu auf den produktiven System zu installieren.} noch einen Schritt weiter gegangen werden. Um dieses Ziel zu erreichen, muss die Architektur der Applikation, mit dem Schwerpunkt Kommunikation zwischen der Teilen, angepasst werden. Des Weiteren sind gegebenefalls Änderungen an Processen wie Change Managemen notwendig.
Der Vorteil dieses Ansatzes erlaubt es, Fehler oder Erweiterungen nach einem standartisierten und automatischen Prozess ohne menschliche Intervention zu installieren. Dadruch kann die Firma sehr schnell auf sich änderne Anforderungen reagieren.


\section{Systemabgrenzung}

Finanzdienstleiter wie die SIX haben für die Abwicklung ihres Geschäfts eine grosse Anzahl Applikation welche verschiedene Geschäftstätikeiten abdecken. Auch das Merchant Onboarding ist ein Teil dieser Landschaft und kann nur durch Interaktion mit diesen System seine Aufgabe erledigen. In der folgenden Tabelle sind kurz die Systeme aufgeführt welche Merchant Onboaring verwendet, jedoch nicht Teil der Problemstellung und Lösung sind.

\begin{table}[H]
	\centering
	\caption{Verwendete Applikationen}
	\begin{tabular}{ | p{2cm} | p{14cm} | }
		\toprule
		{\textbf{Applikation}} & {\textbf{Beschreibung}} \\
		\midrule
		PASS & \textbf{P}ayment \textbf{A}cquiring \textbf{S}ervices \textbf{S}ystem. Applikation welche die Transaktionen verarbeitet und Händlerauszahlungen auslöst. \\ \hline
		Workflow Engine & Business Process Engine welche sich die Erstellung des Kundenkontos kümmert.\\ \hline
		SCS & Registrationstelle für neue Zahlungsterminal.\\ \hline
		ZEFIX & Schnittstelle zum Handelsregister\\ \hline
		AddressService Post\\ \hline
		Google Maps\\ \hline
		ATMS Service (MTAN)\\
		\bottomrule
	\end{tabular}
\end{table}

