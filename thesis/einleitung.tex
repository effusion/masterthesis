\chapter{Einleitung}

\section{Problemstellung}

Die Applikation verwendet bereits Continuous Delivery \footnote{test text} für die Ausrollung auf Testumgebungen. Der nächste Schritt ist jetzt die Applikation mittels Continuous Deployment direkt in den Betrieb zu bringen. Das heisst, dass bei jedem Commit eines Entwicklers, welcher alle automatisierten Tests und Qualitätsüberprüfungen durchläuft, die Applikation Live geht.
Das direkte Ausrollen einer Applikation in den Betrieb benötigt nicht nur technische, sondern auch organisatorische Anpassungen. 
Auf der technischen Seite bedeutet dies, dass die Architektur der Applikation so ausgelegt sein muss, dass eine direkte Ausrollung in den Betrieb, möglichst unterbruchsfrei, machbar ist. Entsprechend muss ein Sicherheitsnetz aus automatisierten Tests (Unit, Integration, Load und Security) sowie Codequalitätsüberprüfungen vorhanden sein resp. aufgebaut werden, damit die Qualität sichergestellt ist. Ein weiterer Punkt ist der ganze Build Prozess welcher die Bereitstellung verschiedener Umgebungen erlaubt ohne das der Entwickler oder ein Systemadministrator manuell eingreifen muss. Dies folgt dem Ansatz von DevOps bei welchem Entwicklung und Betrieb enger zusammenarbeiten. Da die Applikation aus mehreren Teilen besteht, darunter auch ein Webinterface, muss dies entsprechend robust sein, damit der Benutzer nicht merkt, dass eine neue Version der Applikation ausgerollt wurde.
Auf der organisatorischen Seite kommen Richtlinien, Compliance und Gesetze ins Spiel. Da die SIX mit Kreditkarten Informationen arbeitet, muss sie sich an die PCI DSS Richtlinien halten welche den Umgang, die Sicherung und den Schutz der Karteninformationen regelt. Obschon die SIX mittlerweile den DevOps \footnote{Methodik der Zusammenarbeit zwischen der Entwicklung und dem Betrieb von Software verbessert. } Ansatz verfolgt und interne Initiativen gestartet hat, sind auf Grund von Compliance Anforderungen gewisse Richtlinien vorgegeben.

SIX betreibt die schweizerische Finanzplatzinfrastruktur, welche weltweit umfassende Dienstleistungen in den Bereichen Wertschriftenhandel, Börsentransaktionen, Finanzinformationen und Zahlungsverkehr anbietet. SIX besteht aus mehreren Divisionen, welche die vorhergenannten Bereiche abdecken. Ein Grossteil des Zahlungsverkehrs in der Schweiz wird über SIX abgewickelt. 
SIX Payment Services ist die Division, welche sich als Acquirer um das Prozessieren von Kartentransaktionen für Schemen wie VISA, MasterCard, Diners usw. kümmert. Mittelpunkt des händlerseitigen Prozessierens ist PASS(Payment Acquiring Services System), welches die Transaktionen von Zahlungen verarbeitet. Für diese Verarbeitung sind diverse Informationen nötig wie Händlerdaten, Kontennummer, verfügbare Schemen usw. Des Weiteren verfügt PASS über diverse Schnittstellen zu anderen System wie Buchhaltung, SAP usw. PASS verarbeitet pro Tag zwischen 3 - 14 Millionen Transaktionen im Wert von 500 Millionen bis 1 Milliarde Schweizer Franken.
Damit neue Händler Kartenzahlungen akzeptieren konnten war ein aufwändiger manueller Prozess notwendig um die benötigten Daten zu sammeln und im PASS und anderen Umsystemen einzutragen. Im Zuge der Digitalisierung wurde eine neue Applikation, Merchant Onboarding, entwickelt über welche sich Händler selber registrieren können und die Daten automatisch in PASS eingespiesen werden.
Die Applikation baut auf neueren Technologien wie AngularJS, Spring und Docker auf und verwendet Continuous Delivery für die Ausrollung auf interne Systeme.


Im Zuge der Digitalisierung ist es für Finanzdienstleister unerlässlich die angebotenen Dienstleistungen schneller weiterzuentwickeln und der Kundschaft bereitzustellen. Vor Allem bei Online-Diensten wird diese immer wichtiger. Mit der Entwicklung von Paymit war eine Lösung notwendig welche es potenziellen neuen Händler einfach erlaubt sich für den neuen Dienst anzumelden.
Daraus ist die Applikation 'Merchant Onboarding' (MEON) entstanden. 

\section{Systemabgrenzung}

PASS, SARE, SAPO, Workflow-Prozesse

