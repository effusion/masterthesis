\chapter{Projektmanagement}

\section{Methode}

Scrum

\section{Risiken}

\begin{table}[h!]
	\centering
	\caption{Risiken}
	\label{tab:table1}
	\begin{tabular}{ | p{2cm} | p{14cm} | }
		\toprule
		{\textbf{Risiko}} & {\textbf{Beschreibung}} \\
		\midrule
		Richtlininen der Firma & Abklärungen und Vorschläge wie die Richtlinien eingehalten werden können oder angepasst werden müssten. \\ \hline
		Neue Technologien & Einlesen in die neuen Technologien vor Beginn. \\ \hline
		Conway's law & Conway's law ist eine Beziehung zwischen der Firmenorganisation und der Softwarearchitektur. Das Gesetz besagt, dass die Softwarearchitektur eines Systems sich nach der Firma richten. \\ \hline
		Fehlende Requirements & Das Projekt wurde in einer kurzen Zeit umgesetzt und hat deshalb keinen Requirement Engineering Prozess durchlaufen. Dardurch sind weder funktionale noch nicht funktionale Anfoderungen vorhanden. Diese müssen als erstes erfasst werden. \\
		\bottomrule
	\end{tabular}
\end{table}

\section{Ablauf}

10.10.16:
Kickoff Meeting mit Flo
Besprechung der ersten Task
Verifikation mittels Mocks da sich aktuell viel an der Applikation durch andere Projekte wie SARE, SAPO ändern wird
Verifikation auf dem Openshift Cluster
Einbezug der Unternehmensarchitekten aber erst zu einem späteren Zeitpunkt

16.10.16:
Erstellen der einzlenen Task im Jira

17.10.16
Springplanning:
Sprint 1 -> 3 Wochen
Requirements erfassen
Stakeholder definieren
Ziele definieren
Risiken genauer definieren
 