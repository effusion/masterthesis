\chapter{Projektmanagement}

\newgeometry{left=2.5cm, right=2.5cm, bottom=2.5cm, top=2.5cm}
\begin{landscape}
\thispagestyle{empty}

\section{Risiken}
	\begin{table}[h!]
		\centering
		\caption{Risiken}
		\label{tab:table1}
		\begin{tabular}{ | p{2cm} | p{10cm} | p{10cm} | }
			\toprule
			{\textbf{Risiko}} & {\textbf{Beschreibung}} & {\textbf{Massnahmen}} \\
			\midrule
			Richtlininen der Firma & Abklärungen und Vorschläge wie die Richtlinien eingehalten werden können oder angepasst werden müssten. & Die bestehenden Richtlinien werden von der Applikation bereits eingehalten. Die vorgaben bezüglich Change Management müssen frühzeitig angegagen werden da es Einfluss auf den Prozess hat. \\ \hline
			Conway's law & Conway's law ist eine Beziehung zwischen der Firmenorganisation und der Softwarearchitektur. Das Gesetz besagt, dass die Software Architektur eines Systems sich nach der Firma richten. & Obschon die Organisation noch eher klassich aufgebaut ist, sind durch firmenintern Initiationen die Entwicklungsabteilungen nicht so stark an Vorgaben gebunden. Dennoch ist das Risiko im auge zu behalten. \\ \hline
			Fehlende Requirements & Das Projekt wurde in kurzen Zeit umgesetzt und hat deshalb keinen Requirement Engineering Prozess durchlaufen. Dardurch sind weder funktionale noch nicht funktionalen Anfoderungen vorhanden. & Die Requirements nach zu erfassen ist in diesem Fall nicht zielführend sondern verursacht nur imensen Aufwand. Da die Applikation klein ist und einen klaren Business Case hat,  wird dieser entsprechend beschrieben im Software Architektur Dokument\\ \hline
			Neue Requirements & Die Requirements für die Architekturanpassung sind nicht definiert. & Da die Anforderung von der Entwicklungsabteilung getrieben wird, sollen die Basisanforderungen als erstes erfasst werden. Änderungen oder Ergänzungen sollen entsprechend nachdokumentiert werden.\\
			\bottomrule
		\end{tabular}
	\end{table}
\vfill
\raisebox{0pt}{\makebox[\linewidth][r]{\thepage}}
\end{landscape}
\restoregeometry

\section{Ablauf}

10.10.16:
Kickoff Meeting mit Flo
Besprechung der ersten Task
Verifikation mittels Mocks da sich aktuell viel an der Applikation durch andere Projekte wie SARE, SAPO ändern wird
Verifikation auf dem Openshift Cluster
Einbezug der Unternehmensarchitekten aber erst zu einem späteren Zeitpunkt

16.10.16:
Erstellen der einzlenen Task im Jira

17.10.16
Springplanning:
Sprint 1 -> 3 Wochen
Requirements erfassen
Stakeholder definieren
Ziele definieren
Risiken genauer definieren
 