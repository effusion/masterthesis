\chapter{Projektmanagement}

\section{Methode}

Für die Umsetzung der Masterthesis wird Scrum verwendet. Scrum ist eine Agile Vorgehensmethodik welche nur wenige Regeln vorgibt. Die Wahl fiel auf diese Methodik da sie erstens sehr schlank ist und schnelle Feedbackzyklen erlaubt wodurch die Lösung besser auf die Bedürfnisse der Stakeholder abgestimmt werden kann. Des Weiteren können Probleme und Risiken dadurch früher entdeckt werde. Scrum ist in der Abteilung, in welcher die Masterthesis gemacht wird, die Standard Methodik für Software Projekte. Dabei werden nicht nur Entwicklungs- sondern auch, Analyse-, Anforderungs- und Testaufgaben damit abgehandelt. Die Position des Product Owners wird von meinem Betreuer eingenommen. Die Sprintdauer wird auf zwei Wochen angesetzt. Da die Masterthesis eine Einzelarbeit ist, werden andere Scrum Prozesse wie Retrospektive und Planning entsprechen vereinfacht oder ganz weggelassen. Scrum bietet den Rahmen für das Erstellen eines Software Architektur Dokumentes auf der Basis von Arc42.
Prototyping soll verwendet werden um die ausgewählten Lösungsstrategien gegenüber den Anforderungen zu verifizieren und gegebenenfalls anzupassen.  

\section{Risiken}

\begin{table}[h!]
	\centering
	\caption{Risiken}
	\label{tab:table1}
	\begin{tabular}{ | p{2cm} | p{14cm} | }
		\toprule
		{\textbf{Risiko}} & {\textbf{Beschreibung}} \\
		\midrule
		Richtlininen der Firma & Abklärungen und Vorschläge wie die Richtlinien eingehalten werden können oder angepasst werden müssten. \\ \hline
		Neue Technologien & Einlesen in die neuen Technologien vor Beginn. \\ \hline
		Conway's law & Conway's law ist eine Beziehung zwischen der Firmenorganisation und der Softwarearchitektur. Das Gesetz besagt, dass die Softwarearchitektur eines Systems sich nach der Firma richten. \\ \hline
		Fehlende Requirements & Das Projekt wurde in kurzen Zeit umgesetzt und hat deshalb keinen Requirement Engineering Prozess durchlaufen. Dardurch sind weder funktionale noch nicht funktionalen Anfoderungen vorhanden. Diese müssen als erstes erfasst werden.\\ \hline
		Neue Requirements & Die Requirements für die Architekturanpassung sind ebenfalls nicht klar definiert. Da es sich bei dem Thema für die SIX um Neuland handelt, werden diese Anforderungen während der Architekturentwicklung definiert.\\
		\bottomrule
	\end{tabular}
\end{table}

\section{Ablauf}

10.10.16:
Kickoff Meeting mit Flo
Besprechung der ersten Task
Verifikation mittels Mocks da sich aktuell viel an der Applikation durch andere Projekte wie SARE, SAPO ändern wird
Verifikation auf dem Openshift Cluster
Einbezug der Unternehmensarchitekten aber erst zu einem späteren Zeitpunkt

16.10.16:
Erstellen der einzlenen Task im Jira

17.10.16
Springplanning:
Sprint 1 -> 3 Wochen
Requirements erfassen
Stakeholder definieren
Ziele definieren
Risiken genauer definieren
 