\chapter{Lösung}

Notizen zum ablauf:

Wie wurde die Lösung erarbeitet. Das Was ist im SAD obschon gewisse Teile zusammengefasst hier aufgeführt werden müssen.

erarbeiten der Anforderungen und den Qualitätszielen. MEON exisitiert bereits keine alten Anforderungen vorhanden. 

Qualitätsziele und neue Anforderungen nicht einfach definierbar da aktuell gar nicht klar wie das überhaupt ausehen soll. Anforderungen deshalb pro forma mal aufgenommen mit der Möglichkeit diese später anzupassen.
Organisatorischer Teil impact weitgehend unklar da das Projekt dies bezüglich grössere freiheiten genossen hat. 

Rechercen zum Thema continuous deployment betreffend

1. Identifizieren der Teilprobleme

- Pipeline
- Versionierung
- Asynchronität
- Datenspeicherung
- Konfigurationsmanagement

2. Suchen von Teillösungen, Beschreiben der Teillösungen.

Diverse Teilösungen indentifiziert anhand der bis zu diesem Zeitpunkt vorhandenen Requirements -> Qualitätsziele nach wievor nicht komplett klar.
Möglichst grosser Lösungsraum an Teilprobleme um eine gute Sicht auf das zu lösende Problem zu bekommen.


Erstellen der Bewertungskriterien.

3. Erstellen von Lösungsvarianten mittels der Teillösungen.

4. Erstellen eines Bewertungskatalogens mit den Stakeholdern/Betreuer

5. Bewerten der Lösungsvarianten.

6. Erstellen der Prototypen.