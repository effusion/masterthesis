\chapter{Lösung}

\section{Requirements}

Unternehmensarchitekten zu weit weg vom geschene deshalb als Stakeholder weg

\section{Kontextabgrenzung}

Wie wurde die Lösung erarbeitet. Das Was ist im SAD obschon gewisse Teile zusammengefasst hier aufgeführt werden müssen.

erarbeiten der Anforderungen und den Qualitätszielen. MEON exisitiert bereits keine alten Anforderungen vorhanden. 

Qualitätsziele und neue Anforderungen nicht einfach definierbar da aktuell gar nicht klar wie das überhaupt ausehen soll. Anforderungen deshalb pro forma mal aufgenommen mit der Möglichkeit diese später anzupassen.
Organisatorischer Teil impact weitgehend unklar da das Projekt dies bezüglich grössere freiheiten genossen hat. 

\section{Qualitätsziele}

\section{Teilprobleme}

- Pipeline
- Versionierung
- Asynchronität
- Datenspeicherung
- Konfigurationsmanagement

\section{Bewertungskriterien}

Erstellen der Bewertungskriterien.
Bewerungskriterien erstellt und entsprechend aufgeführt.

\section{Teillösungen}

Diverse Teilösungen indentifiziert anhand der bis zu diesem Zeitpunkt vorhandenen Requirements -> Qualitätsziele nach wievor nicht komplett klar.
Möglichst grosser Lösungsraum an Teilprobleme um eine gute Sicht auf das zu lösende Problem zu bekommen.

Prototypen erstellen für die Unklarheiten der ausgewählen Teillösungen.

Bild erstellen für Problem der Datenmigration resp. der unterstützung von zwei gleichzeitig laufenden versionen.
RDBMS -> trigger recursion, migration
NoSQL -> kein schema jedoch gleiches problem

CAP Theorem

MySQL CA, Mongo CP

Beide Datenbankentypen erlauben das Problem zu lösen. RDBMS braucht die ändernung an zwei stellen-> Buch über alle möglichen Variationen. Zyrkuläre Trigger tükisch. Mongo: alle muss in Code gemacht werden. Ermöglich schrittweise migration


Evaluation Graphql
Content Negogiation und Versionierung nicht im Prototyp abgebildet da bereits hinreichend bekannt.
Graphql sehr mühesam für Mutationen nicht die gleiche Konfiguration wie für die Abfragen verwendet werden kann.Übergabe wie MAp was sehr viel casten verursacht.. Library nicht von einer Organisation entwickelt sondern von 36 Entwicklern über Github. Gelegentliche Commits jedoch nicht wirk lebendig. 

Erweitern der Bewerungskriterien anhan der erkenntnisse aus den Prototypen.

\section{Lösungsvariante}

MongoDB, OpenShift Spring Cloud Config, Spring Bus

\section{Bewertung}

http://blogs.atlassian.com/2014/04/practical-continuous-deployment/

https://www.thoughtworks.com/insights/blog/architecting-continuous-delivery

https://www.airpair.com/continuous-deployment/posts/continuous-deployment-for-practical-people

https://www.thoughtworks.com/insights/blog/architecting-continuous-delivery

https://www.innoq.com/de/blog/why-restful-communication-between-microservices-can-be-perfectly-fine/

http://microservices.io/patterns/microservices.html

http://scs-architecture.org/vs-ms.html

http://de.slideshare.net/ewolff/rest-vs-messaging-for-microservices

https://www.symfony.fi/entry/versioning-an-api-in-graphql-vs-rest

https://redis.io/commands/rpoplpush

https://redis.io/topics/replication

http://blog.kubernetes.io/2016/04/configuration-management-with-containers.html

https://docs.mongodb.com/v3.2/replication/

https://dzone.com/articles/netflix-oss-spring-cloud-or-kubernetes-how-about-a

http://stackoverflow.com/questions/35998227/can-i-use-kubernetes-as-service-discovery-in-spring-cloud

https://developers.redhat.com/blog/2016/12/09/spring-cloud-for-microservices-compared-to-kubernetes/

http://de.slideshare.net/ewolff/rest-vs-messaging-for-microservices

https://www.innoq.com/de/blog/why-restful-communication-between-microservices-can-be-perfectly-fine/

https://vimeo.com/144746079

https://spring.io/guides/gs/messaging-redis/

https://seanmcgary.com/posts/how-to-build-a-fault-tolerant-redis-cluster-with-sentinel

https://spring.io/guides/gs/centralized-configuration/

https://wiki.jenkins-ci.org/display/JENKINS/zap+plugin

https://www.digitalocean.com/community/tutorials/how-to-create-a-multi-node-mysql-cluster-on-ubuntu-16-04

https://dev.mysql.com/doc/mysql-utilities/1.5/en/utils-task-autofailover.html

\url{https://docs.openshift.com/online/cli_reference/get_started_cli.html}

https://spring.io/guides/gs/spring-boot-docker/

\url{https://docs.openshift.com/online/cli_reference/get_started_cli.html}

https://github.com/OpenShiftDemos/openshift-cd-demo

http://www.sohamkamani.com/blog/2016/06/30/docker-mongo-replica-set/

\url{https://docs.openshift.com/online/using_images/db_images/mysql.html#using-mysql-replication}

https://blog.openshift.com/openshift-3-demo-part-8-rolling-deployments/

http://docs.spring.io/spring-boot/docs/current/reference/html/howto-database-initialization.html

http://dev.mysql.com/doc/refman/5.7/en/trigger-syntax.html

\url{http://mongodb.github.io/mongo-csharp-driver/2.3/reference/bson/mapping/schema_changes/}

\url{https://www.mongodb.com/blog/post/6-rules-of-thumb-for-mongodb-schema-design-part-1}

\url{https://www.mongodb.com/blog/post/6-rules-of-thumb-for-mongodb-schema-design-part-2}

\url{https://www.mongodb.com/blog/post/6-rules-of-thumb-for-mongodb-schema-design-part-3}

https://github.com/graphql-java/todomvc-relay-java

https://github.com/graphql-java/graphql-java

http://www.sohamkamani.com/blog/2016/06/30/docker-mongo-replica-set/

https://kubernetes.io/docs/user-guide/pods/









