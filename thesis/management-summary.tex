\chapter{Management Summary}

Die vorliegende Masterthesis hat das Ziel eine Software Architektur zu entwerfen welche es erlaubt die darauf basierende Applikation, ohne Unterbruch des Dienstes, in einer produktiven Umgebung vollautomatisch zu installieren.\newline\newline
Neue Anforderungen, welche in immer kürzerer Zeit dem Benutzer zur Verfügungen gestellt werden müssen, sind eine neue Herausforderungen für den Finanzdienstleister SIX und die Motivation dieser Arbeit.\newline\newline
Methodisch orientierte sich das Vorgehen an Standard Prozessen der Firma wie Scrum und Prototyping. Die neuen Anforderungen wurden mit dem Product Owner in Qualtitätsszenarien für die neue Architektur übertragen. Die Problematik wurde zuerst in kleinere Teilprobleme aufgeteilt und für diese in einem weit gefassten Lösungsraum Varianten gesucht. Diese wurden mittels einer mit dem Product Owner erarbeiteten Matrix bewertet und bei Unklarheiten mit Prototypen verifiziert. Die daraus entstandenen Teillösungen sind mit der bestehenden Architektur zusammengeführt und die endgültige Lösung in einem Software Architektur Dokument spezifiziert worden. Das Dokument diente als Vorlage für den finalen Prototypen mit welchem die Qualitätsziele verifiziert wurden.\newline\newline
Die Masterthesis zeigt zum Schluss auf, dass die gesetzten Qualitätsziele und Anforderungen, durch gezielten Einsatz von Methoden und Technologien, umgesetzt werden können und was die Folgen für SIX sind.