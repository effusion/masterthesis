\chapter{Ziele}

Das Ziel der Masterthesis ist die aktuelle Architektur der Applikation MEON so anzupassen, dass die Anwendung kontinuierlich ausgerollt werden kann. Für diese neue Architektur wurden Qualitätsziele definiert, welche als Ziele für die Thesis gelten. Folgend sind die Ziele aufgeführt:

\begin{itemize}
	\item Der Zugriff auf sensitive Daten (PCI) darf nicht möglich sein.
	\item Anpassungen an der Software sollen schnell eingeführt werden können.
	\item Die Applikation soll einfach auf unterschiedlichen Umgebungen installiert werden können.
	\item Der Händler soll bei korrektem Ausfüllen der Daten sich registrieren können.
	\item Konfigurationsänderungen an der Applikation können ohne Unterbruch durchgeführt werden.
	\item Die Applikation soll schnell horizontal skaliert werden können.
\end{itemize}

Die Architektur soll in einem Software Architektur Dokument basierend auf dem Arc42 Template dokumentiert werden.

Die Szenarien zu den Qualitätszielen sind im SAD im Kapitel 10 aufgeführt.
