\chapter{Erkenntnisse}

Scrum ist für das Schreiben von Dokumenten bei der viele kleine Task anfallen eher ungeignet. Vor allem wenn eine Architektur entwickelt werden soll bei dem sich während der Erarbeitung kontinuierlich
neue Erkentnisse ergeben. Gemäss Scrum müsste dies dann neu erfasst werden resp. bräuchte einen Scope change. Dadurch wird die Planung sehr grob und überhaupt nicht mehr schätzbar. Ein iteratives vorgehen ist jedoch unerlässlich. Fraglich ist ob vllt kanban besser wäre resp.

Abstraktionen sind wichtig und nötig. Es kann nicht alles im Detail beschrieben werden. Diese Aufgabe fällt den Entwicklern zu.

Es ist nicht mehr möglich alles selber zu evaluieren. Das zeigt sich vor allem bei Prototypen. Obschon man gewissen Probleme identifizieren kann, müssen Lösungen mit spezialisten geklärt werden. Dies trifft speziel bei grösseren Plattformen Openshift oder bei Hochverfügbaren Datenbanken auf. 

\section{Schlussfolgerung}


\section{Massnahmen}