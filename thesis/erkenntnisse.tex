\chapter{Erkenntnisse}

Während der Masterthesis wurden diverse Erkenntnisse gewonnen welche folgend kurz dargestellt werden.\newline \newline \noindent 
Scrum als gewählte Methode für die Abwicklung der Masterthesis hat sich nur bedingt bewährt. Das iterative Vorgehen und die ständige Kommunikation mit den beteiligten Personen und dem Betreuer haben geholfen die Thesis effizient abzuwickeln und Probleme früh zu erkennen, was von Vorteil war. Als Einzelperson bewähren sich die anderen Aspekte wie Planning und das Vorbereiten von User Stories gemessen am Nutzen nicht. Eine Schwierigkeit ist auch das Erstellen von Schätzungen wenn es um Schreiben von Texten geht. Schlussendlich wurde eine einfach Checkliste verwendet um die Aufgaben zu verfolgen.\newline \newline \noindent 
Der Übergang von der Entwicklerrolle in die Architektenrolle hat gezeigt, dass nicht mehr alles genau ausgearbeitet werden kann. Das Finden des richtigen Detaillevels ist wichtig um die Architektur nicht zu überladen, tendiert man als Entwickler doch dazu die Dinge genau verstehen zu wollen. Obschon es sich um eine Einzelarbeit handelte, musste öfters der Rat von Produktverantwortlichen und Administratoren eingeholt werden. Das Entwerfen einer neuen Architektur hat auch gezeigt, dass ein gewisser Erfahrungsschatz an Methoden und Technologien vorhanden sein muss. Das generelle Interesse an der Informatik und die damit verbundenen Weiterbildungen sind faktisch Pflicht, da sich ohne dies verschiedene Perspektiven auf Probleme gar nicht öffnen. Das Sprechen der verschiedenen Sprachen der Stakeholder ist am Anfang ungewohnt und bedarf Übung, eröffnet aber eine ganzheitliche Sicht auf Problemstellungen.\newline \newline \noindent 
Die wichtigste Erkenntnis ist, dass sich durch methodisches, iteratives Zusammenarbeiten und klare Kommunikation ein auf den ersten Blick komplexes Problem gezielt lösen und dokumentieren lässt.

\chapter{Schlussfolgerung}

Die definierte Architektur hat die Anforderungen und die Qualitätsziele erfüllt und gezeigt, das Continuous Deployment innerhalb von SIX möglich ist. Die gewonnenen technischen Erkenntnisse können nun für Merchant Onboarding verwendet und sogar auf andere Projekte übertragen werden. Zusammen mit den bereits gestartete Veränderungen in der Kultur, die Verwendung von angemessenen Technologien und das Überdecken von alten Prozessen, könnte SIX seinen Kunden in Zukunft neue Funktionen, Fehlerbehebungen und sogar komplett neue Anwendungen in kurzer Zeit zu Verfügung stellen.\newline
Das Software Architektur Dokument, im speziellen das Arc42 Template, hat sich als gutes Mittel bewährt um die Anwendung auf einer mittleren Abstraktionsebene zu beschreiben. Ein Dokument alleine kann aber die Kommunikation der Architektur an die Entwickler nicht ersetzten. Der Architekt muss deshalb die Zusammenarbeit suchen um das Projekt schlussendlich zum Erfolg zu bringen.

\chapter{Massnahmen}
Da die komplette Installation von OpenShift noch nicht abgeschlossen ist, muss die Verteilung der Applikation zu einem späteren Zeitpunkt nochmals geprüft werden. Aktuell gibt es noch diverse offene Punkte und Richtlinien welche geprüft werden müssen bevor Anwendungen produktiv auf der Plattform laufen können. Dazu gehört die ganze Verwaltung von den Basis Docker Images, auf welchen die Container basieren. Viele Applikationen haben durch ihre Geschichte verschiedene Arten wie sie gebaut und ausgerollt werden. Des Weiteren eigenen sich nicht zwingend alle Anwendungen für eine Migration oder brauchen zumindest eine Anpassung der Architektur. Obschon, wie zu Beginn erwähnt, Initiativen für eine bessere Zusammenarbeit zwischen Business, Betrieb und Entwicklung gestartet wurden, konzentriert sich das Wissen aktuell auf wenige Personen. Hier bedarf es die Mitarbeiter zu schulen, damit sie die vielen Vorteile, welche sich aus Continuous Deployment ergeben, auch nutzen können.\newline
Im Generellen sollten die Möglichkeiten der Architektur und Technologien in der Firma gezeigt werden. Viele Abteilungen, darunter auch das Change Management, kennen die aktuellen Möglichkeiten der heutigen Technologien nicht. Obschon einige Personen sich offen zeigen, müssen auch Sie zuerst Hindernisse und alte Prozesse aus dem Weg räumen. Dies ist um einiges einfacher wenn die Ideen nicht nur auf Papier stehen sondern an Prototypen demonstriert werden können.\newline
Mit der Verwendung von Continuous Deployment muss auch der Arbeitsprozess angepasst werden. Viele Entwickler und Administratoren sind sich gewohnt erst nach einer längeren Zeit eine Applikation zu installieren. Da dies nicht mehr auf einen Schlag, sondern kontinuierlich geht, müssen die Aufgaben auch so erfasst werden.