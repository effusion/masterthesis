\chapter{Erkenntnisse}

Während der Masterthesis konnten diverse Erkenntnisse gewonnen werden welche folgend kurz dargestellt werden.\newline\newline

Scrum als gewählte Methode für die Abwicklung der Masterthesis hat sich nur bedingt bewährt. Das iterative Vorgehen und die ständige Kommunikation mit den Beteiligten Personen und dem Betreuer haben geholfen die Thesis effizient abzuwickeln und Probleme früh zu erkennen was von Vorteil war. Als Einzelperson bewähren sich die anderen Aspekte wie Planning und das vorbereiten von User Stories gemessen am Nutzen nicht. Eine Schwierigkeit ist auch das erstellen von Schätzungen wenn es um Schreiben von Text geht. Schlussendlich wurde eine einfach Checkliste verwendet um die Aufgaben zu verfolgen.\newline\newline

Der Übergang von der Entwicklerrolle in die Architektenrolle hat gezeigt, dass nicht mehr alles genau Ausgearbeitet werden kann. Das Finden des richtigen Detaillevels ist wichtig um die Architektur nicht zu überladen tendiert man als Entwickler doch dazu die Dinge genau verstehen zu wollen. Obschon es sich um eine einzel Arbeit handelte, musste öfters der Rat von Experten eingeholt werden, was zeigt, dass Vertrauen und Verlassen in seine Umgebung und deren Fähigkeiten ein wichtiger Bestandteil der Arbeit ist. Das Sprechen der verschiedenen Sprachen der Stakeholder ist am Anfang ungewohnt und bedarf Übung eröffnet aber eine ganzheitlichere Sicht auf Problemstellungen.\newline\newline

Die wichtigste Erkenntnis ist, dass sich durch methodisches iteratives Zusammenarbeiten und klare Kommunikation ein, auf den ersten Blick komplexes Problem, gezielt lösen lässt.

\chapter{Schlussfolgerung}

Die definierte Architektur hat die Anforderungen und die Qualitätsziele erfüllt und gezeigt, dass Continuous Deployment inner halb von SIX möglich ist. Die gewonnen technischen Erkenntnisse können nun für Merchant Onboarding verwendet und sogar auf andere Projekte übertragen werden. Zusammen mit den bereits gestartete Veränderungen in der Kultur, die Verwendung von angemessenen Technologien und das Überdecken von alten Prozessen, kann SIX seinen Kunden in Zukunft neue Funktionen, Fehlerbehebungen und sogar komplett neue Anwendungen in kurzer Zeit zu Verfügung stellen.

\chapter{Massnahmen}

Da die komplette Installation von OpenShift noch nicht abgeschlossen ist, muss die Verteilung der Applikation zu einem späteren Zeitpunkt nochmals geprüft werden. Durch die enge Zusammenarbeit zwischen Entwicklung und Betrieb ist dies jedoch kein grosses Hindernis.