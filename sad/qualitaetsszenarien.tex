\chapter{Qualitätsszenarien}
\label{sec:qualityscenarios}

\section{Qualitätsszenarien}

\subsection{Qualitätsbaum}

\begin{tabular}{l|l|l|c}
Merkmal & Untermerkmal & Szenario & Prio \\ \hline
Funktionalität & Sicherheit & Q1: Der Zugriff auf sensitive Daten (PCI) darf nicht möglich sein & 1 \\
Änderbarkeit & Modifizierbarkeit & Q2: Anpassungen an der Software sollen einfach eingeführt werden können & 2 \\
Funktionalität & Ordnungsmässigkeit & Q3: Nur verifizierte Händler dürfen sich onboarden & 3 \\
Benutzbarkeit & Bedienbarkeit & Q4: Der Händler soll sich mit dem minimalen Set an Informationen onboarden können & 4 \\
Übertragbarkeit & Installierbarkeit & Q5: Die Applikation soll einfach auf unterschiedlichen Umgebungen installiert werden können & 5 \\
Zuverlässigkeit & Fehlertoleranz & Q6: Der Händler soll bei korrektem Ausfüllen der Daten sich registrieren können & 6
\\ \hline
\end{tabular}

\subsection{Bewertungsszenarien}
\subsubsection{Szenario: Der Zugriff auf sensitive Daten (PCI) darf nicht möglich sein}
%% https://www.pcisecuritystandards.org/document_library
Nach PCI Standard müssen Kartendaten geschützt werden und dürfen nicht unberechtigt zugegriffen werden.
\paragraph{}
\begin{tabular}{l|l}
\hline
Geschäftsziel & Damit SIX Kartendaten verarbeiten kann müssen die PCI Standards eingehalten werden. Die Händler werden in einem System mit sensitiven Daten aufgesetzt, auf dieses System soll kein Zugriff möglich sein. \\ \hline
Auslöser & Ein Krimineller versucht an Kartendaten zu gelangen. \\ \hline
Reaktion & Der Zugriff über die Schnittstelle zum System ist nur schreibend und wird überwacht. Unregelmässigkeiten werden geloggt.
\\ \hline
\end{tabular}

\subsubsection{Szenario: Anpassungen an der Software sollen einfach eingeführt werden können}

Es muss möglich sein neue Businessanforderungen und Fehlerbehebungen in der Software schnell nach der Umsetzung in der produktiven Platform zu integrieren. 
\paragraph{}
\begin{tabular}{l|l}
\hline
Geschäftsziel & Die Anforderungen vom Business (z.B. Risk) oder Fehler in der Software muss möglichst zeitnah ausgerollt werden. \\ \hline
Auslöser & Eine Änderung oder ein Fehler führen zu rechtlich inkonsistentem Zustand \\ 
\hline
Reaktion & Die Software wird vom Entwickler behoben und kann nach Abnahme des zuständigen Stakeholders selbstständig in der Produktion ausgerollt werden \\ 
\hline
Zielwerte & Durchlauf von Abnahme bis zur produktiven Benutzbarkeit soll in unter 10 Minuten erreicht werden.
\\ \hline
\end{tabular}

\subsubsection{Szenario: Nur verifizierte Händler dürfen sich onboarden}
Jeder Händlern, der sich über das Portal onboardet, soll so verifiziert sein, dass eine gültige Emailadresse und eine gültige Telefonnummer verfügbar sind.
\paragraph{}
\begin{tabular}{l|l}
\hline
Geschäftsziel & Über das Portal sollen nur potentielle Kunden sich onboarden können. \\ 
\hline
Auslöser & Eine Maschine oder eine Person ohne Interesse an der Dienstleistung versuchen sich im System auzusetzen. \\ 
\hline
Reaktion & Das System stellt sicher, dass nur ein Benutzer mit gülter Emailadresse und Telefonnummer zum Onboardingbereich Zugriff erhalten.
\\ \hline
\end{tabular}

\subsubsection{Szenario: Der Händler soll sich mit dem minimalen Set an Informationen onboarden können}
Acquiringverträge haben > 100 Felder, die ein Händler ausfüllen muss.
\paragraph{}
\begin{tabular}{l|l}
\hline
Geschäftsziel & Der Benutzer soll soweit als möglich unterstützt werden, so dass ein Vertrag mit den minimalen Eingaben vom Benutzer vollständig ausgefüllt werden kann \\
\hline
Auslöser & Der Benutzer will sich onboarden ohne im Büro zu sitzen (Use Case im Taxi am Tablet) \\ 
\hline
Reaktion & Die Oberfläche unterstützt den Benutzer mit Autocompletion / Automatischem Handelsregisterauszug / Fotouploadmöglichkeiten
\\ \hline
\end{tabular}

\subsubsection{Szenario: Die Applikation soll einfach auf unterschiedlichen Umgebungen installiert werden können}
Die Applikation soll sowohl intern für Tester als auch extern für Businessabnahmen oder zur Skalierung einfach in verschiedenen Umgebungen installiert werden können.
\paragraph{}
\begin{tabular}{l|l}
\hline
Geschäftsziel & Die Applikation ist standortunabhängig in verschiedenen Versionen verfügbar. \\
\hline
Auslöser & Ein Businessvertreter möchte die Applikation zur Abnahme zur Verfügung gestellt bekommen. \\ 
\hline
Reaktion & Die Applikation wird in einer neuen Umgebung dem Businessvertreter zugänglich gemacht. 
\\ \hline
\end{tabular}

\subsubsection{Szenario: Der Händler soll bei korrektem Ausfüllen der Daten sich registrieren können}
Das Abschicken der Registrierungsdaten soll bei validem Inhalt jederzeit möglich sein.
\paragraph{}
\begin{tabular}{l|l}
\hline
Geschäftsziel & Der Benutzer füllt das Formular korrekt aus und meldet sich damit bei der SIX als Kunde an. \\
Auslöser & Der Kunde drückt den Registierungsknopf am Ende des Pagewizards \\  \hline
Reaktion & Der Kunde kriegt eine Bestätigung, dass sein Formular bearbeitet wird. \\ \hline
Zielwerte & Unabhängig der Verfügbarkeit der Umsysteme muss ein Kunde erfolgreich die Registrierung abschlissen können
\\ \hline
\end{tabular}


