\graphicspath{{./images/}}

\newcounter{nonFuncReq} %Non functinal requiremet counter
\newcounter{quatar} %Quality target counter

\chapter{Einführung und Ziele}

\section{Business Case}

Die Applikation MEON dient der Registration neuer händler für den mobilen Bezahldienst Twint. 

\section{Aufgabenstellung}

Die Architektur der Applikation MEON soll so angepasst werde dann Continuous Deployment möglich wird ohne das der Benutzer etwas davon merkt. Funktional ist die Anwendung komplett.

\section{Anforderungen}

Alle funktionalen Anforderungen sind bereits umgesetzt. Durch die Anpassung der Architektur Richtung Continuous Deployment gibt es aktuell nur nicht funktionale Anforderungen.

\begin{table}[H]
	\centering
	\caption{Nicht funktionale Anforderungen}
	\begin{tabular}{ | p{2cm} | p{14cm} | }
		\toprule
		{\textbf{ID}} & {\textbf{Beschreibung}} \\
		\midrule
		NFA-\arabic{nonFuncReq} \stepcounter{nonFuncReq} & Kunde merkt nicht wenn neue Komponenten der Applikation ausgetauscht werden. \\ \hline
		NFA-\arabic{nonFuncReq} \stepcounter{nonFuncReq} & Verfügbarkeit der Anwendung ist 24x7 (SLA). \\ \hline
		NFA-\arabic{nonFuncReq} \stepcounter{nonFuncReq} & Einhaltung der PCI DSS Anforderung bezüglich Umgang mit Daten resp. deren Zugriffsschutz durch Authentifizierung, Authorisierung und entsprechende Netzwerksegmentierung. \\ \hline
		NFA-\arabic{nonFuncReq} \stepcounter{nonFuncReq} & Die Applikation muss einfach skalierbar sein \\ \hline
		NFA-\arabic{nonFuncReq} \stepcounter{nonFuncReq} & Die Anwendung muss eine hohe Ausfallsicherheit aufweisen \\ \hline
		NFA-\arabic{nonFuncReq} \stepcounter{nonFuncReq} & Bugfixing mittels Vorwärtscommit. Anstelle von Hotfixes wird immer die ganze Applikation neu ausgerollt. \\ \hline
		NFA-\arabic{nonFuncReq} \stepcounter{nonFuncReq} &  \\ \hline
		NFA-\arabic{nonFuncReq} \stepcounter{nonFuncReq} & Die Anwendung lässt sich voll automatisch Ausrollen. \\
		\bottomrule
	\end{tabular}
\end{table}

\subsection{Aspekte}

Die wichtigstens Aspekte der Fachdomäne sind: 
\begin{itemize}
	\item Auswahl eines Paymit Abos
	\item Vertragsabschluss
\end{itemize}

\subsection{Art des Systems}

Bei MEON handelt es sich um ein interaktives Online system.

\subsection{Nutzung des Systems}

\subsection{Schnittstellen}

MEON hat folgende Schnittstellen zu Umsystemen.
\begin{itemize}
	\item PASS: Transaktionsverarbeitungssystem.
	\item SCS:  Verwaltung der Terminal geräte bei den Kunden.
	\item ZEFIX: Handelsregsiteranbindung.
\end{itemize}

\subsection{Datenhaltung}

\begin{itemize}
	\item RDBMS (Presistenz)
\end{itemize}

\section{Qualitätsziele}

\begin{table}[H]
	\centering
	\caption{Qualitätsziele}
	\begin{tabular}{ | p{3cm} | p{13cm} | }
		\toprule
		{\textbf{Name}} & {\textbf{Beschreibung}} \\
		\midrule
		Q-\arabic{quatar} \stepcounter{quatar} & Die neue Architektur soll einfach verständlich und anwendbar sein.\\ \hline
		Q-\arabic{quatar} \stepcounter{quatar} & Die Deployment Pipeline soll einfach wartbar sein.\\ 
		\bottomrule
	\end{tabular}
\end{table}

\section{Stakeholder}

\begin{table}[H]
	\centering
	\caption{Stakeholder}
	\begin{tabular}{ | p{3cm} | p{13cm} | }
		\toprule
		{\textbf{Name}} & {\textbf{Beschreibung}} \\
		\midrule
		Business & Repräsentiert die Geschäftstätigkeit der Firma.\\ \hline
		Kunde & Benutzer der Webapplikation. \\ \hline
		Support-Kunde & Support welcher den Kunden bei Problemen unterstützt. \\ \hline
		Legal &  Verantwortlich für die Richtigkeit der Vertragsabschlüsse. \\ \hline
		Risk & Sorgt für die Überprüfung der Personen und der Geschäftstätigkeiten. \\ \hline
		Marketing & Abteilung für den Vertrieb, Werbung, Design. \\ \hline
		Betrieb & Betreibt die Webapplikation auf der Infrastruktur der SIX \\ \hline
		Change Management & Genehmigt Änderungen an der Applikation. \\ \hline
		Entwickler & Setzt die Anforderungen der einzelnen Stakeholder um. \\  \hline
		Unternehmens-architekten & Definieren firmenweite Architekturen und haben den gesamtüberblick über die Applikationslandschaf.\\ 
		\bottomrule
	\end{tabular}
\end{table}