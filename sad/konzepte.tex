\chapter{Konzepte}

\section{Fachliche Strukturen (Domäne)}

\begin{quote}
	Fachliche Modelle, Domänenmodelle, Business-Modelle – sie alle beschreiben Strukturen der reinen Fachlichkeit, also ohne Bezug zur Implementierungs- oder Lösungstechnologie.
	Oftmals tauchen Teile solcher fachlichen Modelle an vielen Stellen in der Architektur, insbesondere der Bausteinsicht, wieder auf. 
	Das "Domain-Driven-Design" oder die uralte "essentielle Systemanalyse" können Ihnen hierbei helfen.
	
	Mit Absicht stellen wir diesen Abschnitt an den Anfang der übergreifenden Konzepte.
\end{quote}

\section{Typische Muster und Strukturen}

\subsection{Dependency Injection}

Dieser Mechanismu stellt die korrekte zusammensetztung von Komponenten im Code sicher. Die Abhängigkeite müssen deshalb nicht durch den Entwickler im Code programmiert werden da das Framework diese Aufgabe übernimmt. 

\subsection{Repository}

Repositories sind eine Abstraktionschicht für die Busisses Logik welche den Zugriff auf die Datenbank abstrahiert. Sämtliche Operationen zum Speichern, Lesen und Aktualisieren, sind über Repositories zu tätigen. 

\subsection{Controller}

Controller dienen als Schnittstelle zwischen einem Web Framework und einer Server Applikation. Dabei werden URL Pfad auf einem Contollermethode abbgebildet und können dadurch von einem Web Client angesprochen werden. Für moderen Applikationen wird als Transportformat JSON verwendet. 

\section{Ablaufsteuerung}

\begin{quote}
	Ablaufsteuerung von IT-Systemen bezieht sich sowohl auf die an der (grafischen) Oberfläche sichtbaren Abläufe als auch auf die Steuerung der Hintergrundaktivitäten. Zur Ablaufsteuerung gehört daher unter anderem die Steuerung der Benutzungsoberfläche, die Workflow- oder Geschäftsprozessteuerung sowie Steuerung von Batchabläufen.
\end{quote}

\section{Ausnahme- und Fehlerbehandlung}

\begin{quote}
	Wie werden Programmfehler und Ausnahmen systematisch und konsistent behandelt?
	Wie kann das System nach einem Fehler wieder in einen konsistenten Zustand gelangen? Geschieht dies automatisch oder ist manueller Eingriff erforderlich? Dieser Aspekt hat mit Logging, Protokollierung und Tracing zu tun.
	Welche Art Ausnahmen und Fehler behandelt ihr System? Welche Art Ausnahmen werden an welche Außenschnittstelle weitergeleitet und welche Ausnahmen behandelt das System komplett intern? Wie nutzen Sie die Exception-Handling Mechanismen ihrer Programmiersprache? Verwenden Sie checked- oder unchecked-Exceptions?
\end{quote}


\section{Build}

Jenkins, OC Plugin, Pipleline, Feature Branches

\section{Codegenerierung}

Die AngularJS Services welche die Anfragen an den Server schicken werden anhand von Swagger und den RestController automatisch während dem Build Prozess generiert. Sie können durch den Entwickler auch lokal erstellt werden wenn Änderungen gemacht wurden.

\section{Docker Container}
\label{container}
Container, welche vor allem im Zusammenhang mit Docker stehen, sind Software Einheiten mit eingenem Betreibsystem und können auf einen Host wie Linux gestartet werden. Dabei bekommen die Container einen eigenen Bereich auf dem Host System in welchem sie laufen. Dadurch kann Software als Komplettpakt, gegebenenfalls mit zusätzlichen Bibliotheken, ausgerollt werden ohne auf dem Host-System Änderungen machen zu müssen. \newline
Bevor ein Container gestartet werden kann, muss zuerst eine Definition in Form eines Dockerfiles vorliegen. Aus dieser Definition ensteht ein Image welches danach gestartet werden kann und dann als Container läuft. Container sind unverändlerlich und verlieren alle Daten welche darin gespeichert wurden. Für diesen Fall wurden persistente Volumen geschaffen welche an einen Container angehängt werden können.

\section{Bedienoberfläche}

Die Benutzeroberfläche wurde mittels des Single Page Application Frameworks Angularjs entwickelt um den Benutze

\section{Ergonomie}

\begin{quote}
	Ergonomie von IT-Systemen bedeutet die Verbesserung (Optimierung) deren Benutzbarkeit aufgrund objektiver und subjektiver Faktoren. Im wesentlichen zählen zu ergonomischen Faktoren die Benutzungsoberfläche, die Reaktivität (gefühlte Performance) sowie die Verfügbarkeit und Robustheit eines Systems.
\end{quote}

\section{Feature Toggles}

Toggles erlaubt das Ein- und Ausschalten bestimmter Funkionalität einer Software. Ziel ist es, Änderungen welche an einer Anwendung gemacht wurden, erst zu einem späteren Zeitpunkt zu aktivieren. Für den Anwendungsfall von Continuous Deployment ist dies zwingen notwending da die während der Installattion verschieden Versionen der Software am laufen sind. Neue Funktionen können daher erst aktiviert werden wenn alle Teil der Applikation entsprechend aktualisiert wurden.

\section{Geschäftsregeln}

wird auf der BPM Engine ausgeführt.
\begin{quote}
	Wie behandeln Sie Geschäftslogik oder Geschäftsregeln? Implementieren die beteiligten Fachklassen ihre Logik selbst, oder liegt die Logik in der Verantwortung einer zentralen Komponente? Setzen Sie eine Regelmaschine (rule-engine) zur Interpretation von Geschäftsregeln ein (Produktionsregelsysteme, forward- oder backward-chaining)?
\end{quote}

\section{Hochverfügbarkeit}

DB Replica Set
Komponenten mehrfach vorhanden über PODS

\begin{quote}
	Wie erreichen Sie hohe Verfügbarkeit des Systems? Legen Sie Teile redundant aus? Verteilen Sie das System auf unterschiedliche Rechner oder Rechenzentren? Betreiben Sie Standby-Systeme?
	Könnte in Zusammenhang zu Clusterung stehen.
\end{quote}

\section{Internationalisierung}

\begin{quote}
	Unterstützung für den Einsatz von Systemen in unterschiedlichen Ländern, Anpassung der Systeme an länderspezifische Merkmale. Bei der Internationalisierung (aufgrund der 18 Buchstaben zwischen I und n des englischen Internationalisation auch i18n genannt) geht es neben der Übersetzung von Aus- oder EIngabetexten auch um verwendete Zeichensätze, Orientierung von Schriften am Bildschirm und andere (äußerliche) Aspekte.
\end{quote}

\section{Kommunikation, Integration}

übertragung wie REST zwischen den Services Bus für Konfiguations Änderungen(Siehe Konfiguration)

\begin{quote}
	Kommunikation: Übertragung von Daten zwischen System-Komponenten. Bezieht sich auf Kommunikation innerhalb eines Prozesses oder Adressraumes, zwischen unterschiedlichen Prozessen oder auch zwischen unterschiedlichen Rechnersystemen.
	Integration: Einbindung bestehender Systeme (in einen neuen Kontext). Auch bekannt als: (Legacy) Wrapper, Gateway, Enterprise Application Integration (EAI).
\end{quote}

\section{Konfiguration}

Zweiteilig:
Spring Cloud Config für die Applikations internen Properties.
Deployment über OpenShift Kubernetes

\begin{quote}
	Die Flexibilität von IT-Systemem wird unter anderem durch ihre Konfigurierbarkeit beeinflusst, die Möglichkeit, manche Entscheidungen hinsichtlich der Systemnutzung erst spät zu treffen. Konfiguration kann zu folgenden Zeitpunkten erfolgen:
	Während der Programmierung: Dabei werden beispielsweise Server-, Datei- oder Verzeichnisnamen direkt ("hart") in den Programmcode aufgenommen.
	Während des Deployments oder der Installation: Hier werden Konfigurationsinformationen für eine bestimmte Installation angegeben, etwa der Installationspfad.
	Beim Systemstart: Hier werden Informationen vor oder beim Programmstart dynamisch gelesen.
	Während des Programmablaufs: Konfigurationsinformation wird zur Programmlaufzeit erfragt oder gelesen.
\end{quote}

\section{Logging, Protokollierung}

Splunk

\begin{quote}
	
	Es gibt zwei Ausprägungen der Protokollierung, das Logging und das Tracing . Bei beiden werden Funktions- oder Methodenaufrufe in das Programm aufgenommen, die zur Laufzeit über den Status des Programms Auskunft geben.
	In der Praxis gibt es zwischen Logging und Tracing allerdings sehr wohl Unterschiede:
	\begin{itemize}
		\item Logging kann fachliche oder technische Protokollierung sein, oder eine beliebige Kombination von beidem.
		\item Fachliche Protokolle werden gewöhnlich anwenderspezifisch aufbereitet und übersetzt. Sie dienen Endbenutzern, Administratoren oder Betreibern von Softwaresystemen und liefern Informationen über die vom Programm abgewickelten Geschäftsprozesse.
		\item Technische Protokolle sind Informationen für Betreiber oder Entwickler. Sie dienen der Fehlersuche sowie der Systemoptimierung.
		\item Tracing soll Debugging -Information für Entwickler oder Supportmitarbeiter liefern. Es dient primär zur Fehlersuche und -analyse.
	\end{itemize}
\end{quote}

\section{Management und Administrierbarkeit}

Kein Sepatates UI in der Applikation. Management über OpenShift Web oder OC Client

\begin{quote}
	Größere IT-Systeme laufen häufig in kontrollierten Ablaufumgebungen (Rechenzentren) unter der Kontrolle von Operatoren oder Administratoren ab. Diese Stakeholder benötigen einerseits spezifische Informationen über den Zustand der Programme zur Laufzeit, andererseits auch spezielle Eingriffs- oder Konfigurationsmöglichkeiten.
\end{quote}

\section{Migration}

Da aktuell als Datenbank MySQL verwendet wird, muss eine Migration nach MongoDB durchgeführt werden. Da die Daten von MEON durch den Wechsel von Paymit nach TWINT
obsolet werden und nur wenig Laufdaten gespeichert sind, können die Daten relativ einfach übernommen werden.

\begin{quote}
	Für manche Systeme gibt es existierende Altsysteme, die durch die neuen Systeme abgelöst werden sollen. Denken Sie als Architekt rechtzeitig auch an alle organisatorischen und technischen Aspekte, die zur Einführung oder Migration der Architektur beachtet werden müssen.
	Beispiele:
	\begin{itemize}
		\item Konzept, Vorgehensweise oder Werkzeuge zur Datenübernahme und initialen Befüllung mit Daten
		\item Konzept zur Systemeinführung oder zeitweiliger Parallelbetrieb von Alt- und Neusystem
	\end{itemize}
	Müssen Sie bestehende Daten migrieren? Wie führen Sie die benötigten syntaktischen oder semantischern Transformationen durch?
\end{quote}

\section{Parallelisierung / Nebenläufigkeit}

\begin{quote}
	Programme können in parallelen Prozessen oder Threads ablaufen - was die Notwendigkeit von Synchronisationspunkten mit sich bringt. Die Grundlagen dieses Aspekten legt die Parallelverarbeitung. Für die Architektur und Implementierung nebenläufiger Systeme sind viele technische Detailaspekte zu berücksichtigen (Adressräume, Arten von Synchronisationsmechanismen (Guards, Wächter, Semaphore), Prozesse und Threads, Parallelität im Betriebssystem, Parallelität in virtuellen Maschinen und andere).
\end{quote}

\section{Persistenz}
\label{persistenz}

Mongo DB
Replica Set

\begin{quote}
	Persistenz (Dauerhaftigkeit, Beständigkeit) bedeutet, Daten aus dem (flüchtigen) Hauptspeicher auf ein beständiges Medium (und wieder zurück) zu bringen.
	Einige der Daten, die ein Software-System bearbeitet, müssen dauerhaft auf einem Speichermedium gespeichert oder von solchen Medien gelesen werden:
	\begin{itemize}
		\item Flüchtige Speichermedien (Hauptspeicher oder Cache) sind technisch nicht für dauerhafte Speicherung ausgelegt. Daten gehen verloren, wenn die entsprechende Hardware ausgeschaltet oder heruntergefahren wird.
		\item Die Menge der von kommerziellen Software-Systemen bearbeiteten Daten übersteigt üblicherweise die Kapazität des Hauptspeichers.
		\item Auf Festplatten, optischen Speichermedien oder Bändern sind oftmals große Mengen von Unternehmensdaten vorhanden, die eine beträchtliche Investition darstellen.
	\end{itemize}
	Persistenz ist ein technisch bedingtes Thema und trägt nichts zur eigentlichen Fachlichkeit eines Systems bei. Dennoch müssen Sie sich als Architekt mit dem Thema auseinander setzen, denn ein erheblicher Teil aller Software-Systeme benötigt einen effizienten Zugriff auf persistent gespeicherte Daten. Hierzu gehören praktisch sämtliche kommerziellen und viele technischen Systeme. Eingebettete Systeme (embedded systems ) gehorchen jedoch oft anderen Regeln hinsichtlich ihrer Datenverwaltung.
\end{quote}

\section{Plausibilisierung und Validierung}

Presentation layer durch UID und addresschecker. Handelsregister auszug erneut auf der Workflow enginge
\begin{quote}
	Wo und wie plausibilisieren und validieren Sie (Eingabe-)daten, etwa Benutzereingaben?
\end{quote}

\section{Sessionbehandlung}

Die Applikation besitzt keine Sessions. Eine Anfrage/Bestellung eines Benutzers wird abgesendet und asynchron verarbeitet. Es gibt keine Möglichkeit sich in der Applikation einzulogen.

\section{Sicherheit}

SSL firewall zwischen den Zonen. Zugriffsberechtigung für die Datenbank

\begin{quote}
	Die Sicherheit von IT-Systemen befasst sich mit Mechanismen zur Gewährleistung von Datensicherheit und Datenschutz sowie Verhinderung von Datenmissbrauch.
	Typische Fragestellungen sind:
	\begin{itemize}
		\item Wie können Daten auf dem Transport (beispielsweise über offene Netze wie das Internet) vor Missbrauch geschützt werden?
		\item Wie können Kommunikationspartner sich gegenseitig vertrauen?
		\item Wie können sich Kommunikationspartner eindeutig erkennen und vor falschen Kommunikationspartner schützen?
		\item Wie können Kommunikationspartner die Herkunft von Daten für sich beanspruchen (oder die Echtheit von Daten bestätigen)?
	\end{itemize}
	Das Thema IT-Sicherheit hat häufig Berührung zu juristischen Aspekten, teilweise sogar zu internationalem Recht.
\end{quote}

\section{Skalierung / Clusterung}

Wie bereits im Kapitel \ref{deploy} angesprochen, können auf der OpenShift Plattform die Anzahl Pods eines Services dynamisch angepasst werden. Dies kann über das WebInterface den OC Client oder über Autoscaling geschehen. Da die Applikation nur wenige tägliche Benutzer hat, wird die Skalierung manuell durchgeführt.

\section{Transaktionsbehandlung}

BASE für MongoDB

\begin{quote}
	Transaktionen sind Arbeitsschritte oder Abläufe, die entweder alle gemeinsam oder garnicht durchgeführt werden. Der Begriff stammt aus den Datenbanken - wichtiges Stichwort hier sind ACID-Transaktionen (atomar, consistent, isolated, durable). Im Bereich von NoSQL-Datenbanken gelten andere Kriterien.
\end{quote}

\section{Verteilung}

\begin{quote}
	Verteilung: Entwurf von Software-Systemen, deren Bestandteile auf unterschiedlichen und eventuell physikalisch getrennten Rechnersystemen ablaufen.
	Zur Verteilung gehören Dinge wie der Aufruf entfernter Methoden (remote procedure call, RPC), die Übertragung von Daten oder Dokumenten an verteilte Kommunikationspartner, die Wahl passender Interaktionsstile oder Nachrichtenaustauschmuster (etwa: synchron / asynchron, publish- subsribe, peer-to- peer).
\end{quote}
