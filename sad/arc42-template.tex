\documentclass[11pt,DIV=12]{scrreprt}

\usepackage{layout}

\makeglossaries

\newglossaryentry{REST}
{
	name=REST,
	description={\textbf{Re}presentational \textbf{S}tate \textbf{T}ransfer ist ein Programmierparadigma was sich häufig in Web Applikationen wiederfindet und generell das HTTP Protokoll verwendet.}
}

\newglossaryentry{BPM}
{
	name=BPM,
	description={\textbf{B}usiness \textbf{P}rocess \textbf{E}ngine ist eine Software, Bibliothek welche es erlaubt Geschäftsprozesse zu definieren und auszuführen.}
}

\newglossaryentry{RDBMS}
{
	name=RDBMS,
	description={\textbf{R}elational \textbf{D}ata\textbf{b}ase \textbf{M}anagement \textbf{S}ystem sind Datenbank Systeme welche sich an den relationel Prinzipien orientieren.}
}

\newglossaryentry{NoSQL}
{
	name=NoSQL,
	description={\textbf{N}ot only \textbf{SQL} sind Datenbanken welche sich nicht mehr an die relationalen Prinzipien halten und dadurch alternative Speicher formen für Daten ermöglichen. Die bekanntesten Typen sind Key-Value, Spalten-Orientiert, Dokumente-Orientiert und Graphen. NoSQL Datenbank besitzen im Vergleich zu relationalen Datenbanken kein Schema.}
}

\newglossaryentry{ACID}
{
	name=ACID,
	description={\textbf{A}tomicity \textbf{C}onsistency \textbf{I}solation \textbf{D}urability sind merkmale von relationalen Datenbanken welche stark auf Datenkonsitzenz setzen welches vorallem für Systeme im Finanzbereich unerlässlich sind.}
}

\newglossaryentry{BASE}
{
	name=BASE,
	description={\textbf{B}asically \textbf{A}vailable \textbf{S}oft-State \textbf{E}ventual Consitency ist ein Paradigma welches bei NoSQL Datenbanken anwendung findet. Die Konsitzenz wird dabei gegen vertikal Skalierbarkeit und Partitionierung getauscht.}
}

\newglossaryentry{Container}
{
	name=Container,
	description={Container sind in sich geschlossene Pakete welche ein Betriebsystem und Software enthalten. Sie laufen isololiert auf einem Hostsystem und haben keinen Zugriff auf dieses. Durch die Paketierung können die Container auf verschiedenen Linux Betriebsysteme laufen ohne das darunter liegende System zu beinflussen.}
}

\begin{document}

\graphicspath{{./images/}}

\begin{titlepage}
	\raggedleft
	\includegraphics[width=0.25\textwidth]{sixlogo.png}\par\vspace{1cm}
	\raggedright
	\vspace{2cm}
	{\huge\bfseries Software Architektur Dokument\par}
	\vspace{1cm}
	{\huge Merchant Onboarding\par}
	\vspace{1cm}
	{\large \today\par}
	\vspace{0.5cm}
	{\large Andreas Heubeck: andreas.heubeck@six-groaymitup.com\par}
	\vfill
	\vspace{8cm}
	\textsc{Dieses Dokument basiert auf der Arc42 Architekturvorlage. Erstellt von Dr. Peter Hruschka \& Dr. Gernot Starke. Für weitere Informationen siehe \url{http://www.arc42.de/index.html}}\par
	\vfill
\end{titlepage}

\tableofcontents

\input{abstract}

\chapter{Einführung und Ziele}



\section{Aufgabenstellung}

Die die bestehende Merchant Onboarding Applikation soll eine neue Architektur entwickelt werden welche Continuous Deployment erlaubt.

\section{Qualitätsziele}

\section{Stakeholder}

\begin{table}[H]
	\centering
	\caption{Stakeholder}
	\begin{tabular}{ | p{2cm} | p{14cm} | }
		\toprule
		{\textbf{Name}} & {\textbf{Beschreibung}} \\
		\midrule
		Business & Repräsentiert die Geschäftstätigkeit der Firma.\\ \hline
		Kunde & Benutzer der Webapplikation. \\ \hline
		Support-Kunde & Support welcher den Kunden bei Problemen unterstützt. \\ \hline
		Legal &  Verantwortlich für die Richtigkeit der Vertragsabschlüsse. \\ \hline
		Risk & Sorgt für die Überprüfung der Personen und der Geschäftstätigkeiten. \\ \hline
		Communications & Abteilung für den Vertrieb, Werbung, Design. \\ \hline
		Betrieb & Betreibt die Webapplikation auf der Infrastruktur der SIX \\ \hline
		Change Management & Genehmigt Änderungen an der Applikation. \\ \hline
		Entwickler & Setzt die Anforderungen der einzelnen Stakeholder um. \\ 
		\bottomrule
	\end{tabular}
\end{table}

\chapter{Ziele}

Das Ziel der Masterthesis ist die aktuelle Architektur der Applikation MEON so anzupassen, dass die Anwendung kontinuierlich ausgerollt werden kann. Für diese neue Architektur wurden Qualitätsziele definiert, welche als Ziele für die Thesis gelten. Folgend sind die Ziele aufgeführt:

\begin{itemize}
	\item Der Zugriff auf sensitive Daten (PCI) darf nicht möglich sein.
	\item Anpassungen an der Software sollen schnell eingeführt werden können.
	\item Die Applikation soll einfach auf unterschiedlichen Umgebungen installiert werden können.
	\item Der Händler soll sich bei korrektem Ausfüllen der Daten registrieren können.
	\item Konfigurationsänderungen an der Applikation können ohne Unterbruch durchgeführt werden.
	\item Die Applikation soll schnell horizontal skaliert werden können.
\end{itemize}
Die Architektur soll in einem Software Architektur Dokument basierend auf dem Arc42 Template dokumentiert werden. Die Szenarien zu den Qualitätszielen sind im SAD im Kapitel 10 aufgeführt.


\chapter{Risiken}

\begin{table}[h!]
	\centering
	\caption{Risiken}
	\label{tab:table1}
	\begin{tabular}{ | p{2cm} | p{14cm} | }
		\toprule
		{\textbf{Risiko}} & {\textbf{Beschreibung}} \\
		\midrule
		Richtlinine der Firma & Abklärungen und Vorschläge wie die Richtlinien eingehalten werden können oder angepasst werden müssten. \\ \hline
		Neue Technologien & Einlesen in die neuen Technologien vor Beginn. \\ \hline
		Conway's law & Conway's law ist eine Beziehung zwischen der Firmenorganisation und der Softwarearchitektur. Das Gesetz besagt, dass die Softwarearchitektur eines Systems sich nach der Firma richten. \\
		\bottomrule
	\end{tabular}
\end{table}

\input{ergebnis}

\input{schlussfolgerung}






\end{document}