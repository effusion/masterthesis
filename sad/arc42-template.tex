\documentclass[11pt,DIV=12]{scrreprt}

\usepackage{layout}

\makeglossaries

\newglossaryentry{REST}
{
	name=REST,
	description={\textbf{Re}presentational \textbf{S}tate \textbf{T}ransfer ist ein Programmierparadigma was sich häufig in Web Applikationen wiederfindet und generell das HTTP Protokoll verwendet. Siehe auch \url{https://en.wikipedia.org/wiki/Representational_state_transfer}}
}

\newglossaryentry{BPM}
{
	name=BPM,
	description={\textbf{B}usiness \textbf{P}rocess \textbf{E}ngine ist eine Software Bibliothek welche es erlaubt Geschäftsprozesse zu definieren und auszuführen.}
}

\newglossaryentry{RDBMS}
{
	name=RDBMS,
	description={\textbf{R}elational \textbf{D}ata\textbf{b}ase \textbf{M}anagement \textbf{S}ystem sind Datenbank Systeme welche sich an den relationalen Prinzipien orientieren.}
}	

\newglossaryentry{NoSQL}
{
	name=NoSQL,
	description={\textbf{N}ot only \textbf{SQL} sind Datenbanken welche sich nicht mehr an die relationalen Prinzipien halten und dadurch alternative Speicherformen für Daten ermöglichen. Die bekanntesten Typen sind Key-Value, Spalten-Orientiert, Dokumente-Orientiert und Graphen. NoSQL Datenbank besitzen im Vergleich zu relationalen Datenbanken kein Schema.}
}

\newglossaryentry{ACID}
{
	name=ACID,
	description={\textbf{A}tomicity \textbf{C}onsistency \textbf{I}solation \textbf{D}urability sind Merkmale von relationalen Datenbanken welche stark auf Datenkonsistenz setzen welches vor allem für Systeme im Finanzbereich unerlässlich sind.}
}

\newglossaryentry{BASE}
{
	name=BASE,
	description={\textbf{B}asically \textbf{A}vailable \textbf{S}oft-State \textbf{E}ventual Consitency ist ein Paradigma welches bei NoSQL Datenbanken Anwendung findet. Die Konsistenz wird dabei gegen vertikale Skalierbarkeit und Partitionierung getauscht.}
}

\newglossaryentry{CAP}
{
	name=CAP-Theorem,
	description={\textbf{C}onsitency \textbf{A}vailability \textbf{P}artition Tolerance Theorem ist eine Klassifizierung für verteilte Systeme. Die Kernaussage ist, dass ein System nur zwei der drei Charakteristiken gleichzeitig anbieten kann. }
}

\newglossaryentry{PEP}
{
	name=PEP,
	description={\textbf{P}oliticaly \textbf{E}xposed \textbf{P}erson ist eine Person welche ein politisches Amt ausübt und deshalb in der Öffentlichkeit bekannt ist. Für diese Personen ist eine spezielle Prüfung notwendig.}
}

\newglossaryentry{2PC}
{
	name=Two-Phase Commit,
	description={Der Two-Phase Commit wird vor allem bei transaktionalen Systemen, welche über mehrere System integriert sind, verwendet. Dabei müssen beide System den Abschluss Ihrer Aufgabe bestätigen und nur dann wird auch das Ganze als erfolgreich erachtet. Dieses Konzept findet immer dann Anwendung wenn hohe Anforderungen an die Verfügbarkeit und die Konsistenz gestellt werden und die Performanz zweitrangig ist.}
}

\newglossaryentry{JSON}
{
	name=JSON,
	description={\textbf{J}ava\textbf{S}cript \textbf{O}bject \textbf{N}notation ist ein einfach lesbares Austauschformat welches für den Transport von Daten zwischen Applikation verwendet werden kann.}
}

\newglossaryentry{AMQP}
{
	name=AMQP,
	description={\textbf{A}dvanced \textbf{M}essaging \textbf{Q}ueuing \textbf{P}rotocol dient zur asynchronen Übertragung von Nachrichten über eine Queue und ist seit Mai 2014 ein ISO Standard. }
}

\newglossaryentry{FEBA}
{
	name=Feature Branches,
	description={Werden vor allem bei GIT Repositories verwendet um die Arbeiten der einzelnen Entwicklern voneinander zu trennen und grosse Mergekonflikte zu verhindern.}
}

\newglossaryentry{DDoS}
{
	name=DDoS,
	description={\textbf{D}istributed \textbf{D}enial \textbf{O}f \textbf{S}ervice sind gross angelegte Angriffe über das Internet mit dem Ziel den Service zu überlasten und für andere Benutzer unzugänglich zu machen.}
}

\newglossaryentry{YAML}
{
	name=YAML,
	description={YAML ist eine XML ähnliche Beschreibungssprache welche für Konfigurationsdateien verwendet wird.}
}

\newglossaryentry{URL}
{
	name=URL,
	description={\textbf{U}niform \textbf{R}esource \textbf{L}ocator identifiziert eine Ressource wie zum Beispiele eine Webseite.}
}

\newglossaryentry{MTAN}
{
	name=MTAN,
	description={Stammt von TAN(Transaktionsnummer) ist ein Einmalkennwort welches via SMS verschickt wird. }
}

\begin{document}

\graphicspath{{./images/}}

\begin{titlepage}
	\raggedleft
	\includegraphics[width=0.25\textwidth]{sixlogo.png}\par\vspace{1cm}
	\raggedright
	\vspace{2cm}
	{\huge\bfseries Software Architektur Dokument\par}
	\vspace{1cm}
	{\huge Merchant Onboarding\par}
	\vspace{1cm}
	{\large \today\par}
	\vspace{0.5cm}
	{\large Andreas Heubeck: andreas.heubeck@six-groaymitup.com\par}
	\vfill
	\vspace{8cm}
	\textsc{Dieses Dokument basiert auf der Arc42 Architekturvorlage. Erstellt von Dr. Peter Hruschka \& Dr. Gernot Starke. Für weitere Informationen siehe \url{http://www.arc42.de/index.html}}\par
	\vfill
\end{titlepage}

\tableofcontents

\input{abstract}

\chapter{Einführung und Ziele}



\section{Aufgabenstellung}

Die die bestehende Merchant Onboarding Applikation soll eine neue Architektur entwickelt werden welche Continuous Deployment erlaubt.

\section{Qualitätsziele}

\section{Stakeholder}

\begin{table}[H]
	\centering
	\caption{Stakeholder}
	\begin{tabular}{ | p{2cm} | p{14cm} | }
		\toprule
		{\textbf{Name}} & {\textbf{Beschreibung}} \\
		\midrule
		Business & Repräsentiert die Geschäftstätigkeit der Firma.\\ \hline
		Kunde & Benutzer der Webapplikation. \\ \hline
		Support-Kunde & Support welcher den Kunden bei Problemen unterstützt. \\ \hline
		Legal &  Verantwortlich für die Richtigkeit der Vertragsabschlüsse. \\ \hline
		Risk & Sorgt für die Überprüfung der Personen und der Geschäftstätigkeiten. \\ \hline
		Communications & Abteilung für den Vertrieb, Werbung, Design. \\ \hline
		Betrieb & Betreibt die Webapplikation auf der Infrastruktur der SIX \\ \hline
		Change Management & Genehmigt Änderungen an der Applikation. \\ \hline
		Entwickler & Setzt die Anforderungen der einzelnen Stakeholder um. \\ 
		\bottomrule
	\end{tabular}
\end{table}

\chapter{Ziele}

Das Ziel der Masterthesis ist die aktuelle Architektur der Applikation MEON so anzupassen, dass die Anwendung kontinuierlich ausgerollt werden kann. Für diese neue Architektur wurden Qualitätsziele definiert, welche als Ziele für die Thesis gelten. Folgend sind die Ziele aufgeführt:

\begin{itemize}
	\item Der Zugriff auf sensitive Daten (PCI) darf nicht möglich sein.
	\item Anpassungen an der Software sollen schnell eingeführt werden können.
	\item Die Applikation soll einfach auf unterschiedlichen Umgebungen installiert werden können.
	\item Der Händler soll sich bei korrektem Ausfüllen der Daten registrieren können.
	\item Konfigurationsänderungen an der Applikation können ohne Unterbruch durchgeführt werden.
	\item Die Applikation soll schnell horizontal skaliert werden können.
\end{itemize}
Die Architektur soll in einem Software Architektur Dokument basierend auf dem Arc42 Template dokumentiert werden. Die Szenarien zu den Qualitätszielen sind im SAD im Kapitel 10 aufgeführt.


\chapter{Risiken}

\begin{table}[h!]
	\centering
	\caption{Risiken}
	\label{tab:table1}
	\begin{tabular}{ | p{2cm} | p{14cm} | }
		\toprule
		{\textbf{Risiko}} & {\textbf{Beschreibung}} \\
		\midrule
		Richtlinine der Firma & Abklärungen und Vorschläge wie die Richtlinien eingehalten werden können oder angepasst werden müssten. \\ \hline
		Neue Technologien & Einlesen in die neuen Technologien vor Beginn. \\ \hline
		Conway's law & Conway's law ist eine Beziehung zwischen der Firmenorganisation und der Softwarearchitektur. Das Gesetz besagt, dass die Softwarearchitektur eines Systems sich nach der Firma richten. \\
		\bottomrule
	\end{tabular}
\end{table}

\input{ergebnis}

\input{schlussfolgerung}






\end{document}