\chapter{Risiken}

\section{Risiken}

\begin{table}[h!]
	\centering
	\caption{Risiken}
	\begin{tabular}{ | p{1cm} | p{3cm} | p{10cm} |}
		\toprule
		{\textbf{ID}} & {\textbf{Risiko}} & {\textbf{Beschreibung} } \\
		\midrule
		R1 & OpenShift & Die Container Plattform wird von SIX als strategisches Ziel für eine moderne IT Infrastruktur gesehen. Es gibt bereits eine Testinstanz das endgültige Deployment ist jedoch noch nicht klar was hauptsächlich an Abklärungen bezüglich Compliance liegt.  \\ \hline
		R2 & Mongo & MongoDB ist eine Datebank welche bisher nicht in Betrieb ist und für welches die Firma kein Know-how besitzt.\\
		\bottomrule
	\end{tabular}
\end{table}

\section{Massnahmen}
\begin{table}[h!]
	\centering
	\caption{Massnahmen}
	\begin{tabular}{ | p{1cm} | p{3cm} | p{10cm} | }
		\toprule
		{\textbf{ID}} & {\textbf{Massnahmen}} & {\textbf{Beschreibung}} \\
		\midrule
		R1 & OpenShift & Enge Kommunikation mit den Verantwortlichen welche die Plattform aufbauen. Ein Restrisiko bleibt bestehen welches jedoch nur deployment Anpassungen zu Folge hätten. \\ \hline
		R2 & Mongo & Im Zuge der Evaluation wurden Prototypen gemacht um die Verwendung der Datenbank zu evaluieren. Ein Restrisiko bleibt ist jedoch gering da die Datenbank weltweit bereits bei grösseren Konzernen im Einsatz ist und sich etabliert hat. \\
		\bottomrule
	\end{tabular}
\end{table}