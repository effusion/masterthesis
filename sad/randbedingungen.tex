\chapter{Randbedingungen}

\section{Technische Randbedingungen}

\begin{table}[H]
	\centering
	\caption{Technische Rahmenbedienungen}
	\begin{tabular}{ | p{4cm} | p{12cm} | }
		\toprule
		{\textbf{Technologie / Tool}} & {\textbf{Einsatzzweck}} \\
		\midrule
		Java 8 & Die Serverapplikation ist damit implementiert und wird deshalb wieder verwendet. \\ \hline
		Java Script & Der Webapplikation ist damit implementiert und wird deshalb wieder verwendet. \\ \hline
		Eclipse, Intellij & Entwicklungsumgebung für das Erstellen der Prototypen. \\ \hline
		Spring Framework & Dient als Basisframework für den Serverteil der Applikation.  \\ \hline
		AngularJS & Webframework für die Webapplikation. \\ \hline
		Docker & Deployment der einzelnen Teile der Applikation als Container. \\ \hline
		JBoss 10 & Applikations Server welcher die Laufzeitumgebung für die Applikation bereitstellt. \\ \hline
		Apache Webserver & HTTP Server für den statischen Inhalt. \\ \hline
		Camunda BPM & Workflow-Management-System welches die Prozesse für die Anmeldung eines neuen Händlers steuert. \\ \hline
		RDBMS: Oracle, MySQL / NoSQL: MongoDB & Speichern der Daten. \\ \hline
		SIX Entwicklungs Umgebung  & Verfolgen der Aufgaben, Ablage des Sourcecodes, Code Reviews, Buildserver.\\
		\bottomrule
	\end{tabular}
\end{table}

\section{Organisatorische Randbedingungen}

\begin{table}[H]
	\centering
	\caption{Organisatorische Rahmenbedienungen}
	\begin{tabular}{ | p{4cm} | p{12cm} | }
		\toprule
		{\textbf{Technologie / Tool}} & {\textbf{Einsatzzweck}} \\
		\midrule
		PCI DDS & Richtlinien welche den Umgang mit Kreditkarten regeln \\ \hline
		Change Management & Prozess für das Einführen von Änderungen an einer Applikation\\ 
		\bottomrule
	\end{tabular}
\end{table}

\section{Konventionen}