\chapter{Randbedingungen}

\section{Technische Randbedingungen}

\begin{table}[H]
	\centering
	\caption{Technische Rahmenbedienungen}
	\begin{tabular}{ | p{4cm} | p{11cm} | }
		\toprule
		{\textbf{Technologie / Tool}} & {\textbf{Einsatzzweck}} \\
		\midrule
		Java 8 & Basis Programmiersprache des Onboarding Servers. \\ \hline
		Java Script & Basis Programmiersprache der Webapplikation.\\ \hline
		Eclipse, Intellij & Entwicklungsumgebung für das Erstellen der Prototypen. \\ \hline
		Spring Boot & Basisframework für den Onboarding Server.  \\ \hline
		AngularJS & Webframework für die Webapplikation. \\ \hline
		Docker / OpenShift & Deployment der einzelnen Teile der Applikation als Container. \\ \hline
		Apache Webserver & HTTP Server für den statischen Inhalt der Webseite. \\ \hline
		Camunda BPM & Workflow-Management-System welches die Prozesse für die Anmeldung eines neuen Händlers steuert. \\ \hline
		Oracle 12c & Speichert die Daten der Workflow-Engine \\ \hline
		MongoDB & Speichert die Daten des Onboarding Servers. \\ \hline
		Red Hat Enterprise Linux & Basis Betriebssystem für Server und Container. \\ \hline
		SIX Entwicklungs Umgebung  & Verfolgen der Aufgaben, Ablage des Sourcecodes, Code Reviews, Buildserver.\\
		\bottomrule
	\end{tabular}
\end{table}

\section{Organisatorische Randbedingungen}

\begin{table}[H]
	\centering
	\caption{Organisatorische Rahmenbedienungen}
	\begin{tabular}{ | p{4cm} | p{11cm} | }
		\toprule
		{\textbf{Randbedinung}} & {\textbf{Einsatzzweck}} \\
		\midrule
		\textit{\gls{PCI}} DSS & Richtlinien welche den Umgang mit Kreditkartendaten regeln. \\ \hline
		Change Management & Prozess für das Einführen von Änderungen an einer Applikation.\\ \hline
		Scrum & Standard Vorgehensmodell für die Software Entwicklung.\\ 
		\bottomrule
	\end{tabular}
\end{table}

\section{Konventionen}

\begin{table}[H]
	\centering
	\caption{Konventionen}
	\begin{tabular}{ | p{4cm} | p{11cm} | }
		\toprule
		{\textbf{Konvention}} & {\textbf{Einsatzzweck}} \\
		\midrule
		Coderichtlinien & Für die Java Entwicklung existieren Richtlininen welche mittels Checkstyle und Sonar überprüft werden. \\ \hline
		Sprache & Dokumentation im Code ist Englisch. Für technische Dokumentation kann Englisch oder Deutsch verwendet werden.\\ \hline
		Code Reviews & Für alle  Änderungen wird eine Review durchgeführt. Je nach Risiko nur im Team oder unter Einbezug von Experten.\\ \hline
		Issue Tracking & Für alle Anpassungen muss ein Jira Ticket vorhanden sein. Die Ticketnummer wird  immer in der Commitmessage referenziert.\\	
		\bottomrule
	\end{tabular}
\end{table}