\chapter{Lösungsstrategie}

\section{Teilprobleme}	

Um eine Software Architektur zu entwicklen welche die gegebenen Anforderungen und Qualitätsziele erfüllten, müssen zuerst die einzelnen Teilprobleme indentifiziert und entsprecheden Lösungen gesucht werden.

\begin{table}[H]
	\centering
	\caption{Teilprobleme}
	\begin{tabular}{ | p{4cm} | p{12cm} | }
		\toprule
		{\textbf{Name}} & {\textbf{Beschreibung}} \\
		\midrule
		Schnittstellen Versionierung & Die Software vor Allem die Schnittstellen ändern sich mit der Zeit. Wird die Applikation nun sehr häufig ausgerollt, muss ein Lösung gefunden werden welche es ermöglicht, dass Schnittstellen einfach versioniert werden können.\\ \hline
		Kommunikationsentkopplung &  Durch die kontinuierliche Installation neuer Versionen der Applikation respektiv deren Teile, muss sichergestellt werden, das keine Anfragen verloren gehen.\\ \hline
		Datenspeicherung &  Die Speicherung erfolgt aktuell in einem relationalen Datenbank System. Es muss überprüft werden ob das System mit ständiger Aktualisierung zurecht kommt, oder ob eine andere Lösung gesucht werden muss.\\ \hline
		Konfigurations Management & Das aktuelle Konfiguration Management über Property files oder Docker Variablen ist für die neue Architektur nicht mehr angemessen. Für die Umsetzung braucht einen Ansatz welcher das setzen automatisch und zentral ermöglicht. \\ \hline
		Deployment Pipeline & Die Jenkins Pipeline kann bereits mehere Instanzen ausrollen jedoch muss im Zuge von Continuous Deployment eine Lösung gefunden werden welche diesen Prozess ohne unterbruch des Services erlaubt. \\
		\bottomrule
	\end{tabular}
\end{table}

\section{Architekturentscheide}

Anhand der nicht funktionalen Anforderungen und den Qualitätszielen, wurden folgende architekturrelevanten Entscheidungen getroffen.

\begin{table}[H]
	\centering
	\caption{Architekturentscheide}
	\begin{tabular}{ | p{4cm} | p{12cm} | }
		\toprule
		{\textbf{Qualitätsziel}} & {\textbf{Entscheidung}} \\
		\midrule
		Funktionalität, Sicherheit &  \\ \hline
		Änderbarkeit, Modifizierbarkeit &  \\ \hline
		Übertragbarkeit, Installierbarkeit &  \\ \hline
		Zuverlässigkeit, Fehlertoleranz &  \\
		\bottomrule
	\end{tabular}
\end{table}

http://blogs.atlassian.com/2014/04/practical-continuous-deployment/

https://www.thoughtworks.com/insights/blog/architecting-continuous-delivery

https://www.airpair.com/continuous-deployment/posts/continuous-deployment-for-practical-people

https://www.thoughtworks.com/insights/blog/architecting-continuous-delivery

https://www.innoq.com/de/blog/why-restful-communication-between-microservices-can-be-perfectly-fine/

http://microservices.io/patterns/microservices.html

http://scs-architecture.org/vs-ms.html

http://de.slideshare.net/ewolff/rest-vs-messaging-for-microservices

https://www.symfony.fi/entry/versioning-an-api-in-graphql-vs-rest

https://redis.io/commands/rpoplpush

https://redis.io/topics/replication

http://blog.kubernetes.io/2016/04/configuration-management-with-containers.html

https://docs.mongodb.com/v3.2/replication/

https://dzone.com/articles/netflix-oss-spring-cloud-or-kubernetes-how-about-a

http://stackoverflow.com/questions/35998227/can-i-use-kubernetes-as-service-discovery-in-spring-cloud

https://developers.redhat.com/blog/2016/12/09/spring-cloud-for-microservices-compared-to-kubernetes/

http://de.slideshare.net/ewolff/rest-vs-messaging-for-microservices

https://www.innoq.com/de/blog/why-restful-communication-between-microservices-can-be-perfectly-fine/

https://vimeo.com/144746079

https://spring.io/guides/gs/messaging-redis/

https://seanmcgary.com/posts/how-to-build-a-fault-tolerant-redis-cluster-with-sentinel

https://spring.io/guides/gs/centralized-configuration/

https://wiki.jenkins-ci.org/display/JENKINS/zap+plugin

https://www.digitalocean.com/community/tutorials/how-to-create-a-multi-node-mysql-cluster-on-ubuntu-16-04

https://dev.mysql.com/doc/mysql-utilities/1.5/en/utils-task-autofailover.html

\url{https://docs.openshift.com/online/cli_reference/get_started_cli.html}

https://spring.io/guides/gs/spring-boot-docker/

\url{https://docs.openshift.com/online/cli_reference/get_started_cli.html}

https://github.com/OpenShiftDemos/openshift-cd-demo

http://www.sohamkamani.com/blog/2016/06/30/docker-mongo-replica-set/

\url{https://docs.openshift.com/online/using_images/db_images/mysql.html#using-mysql-replication}

https://blog.openshift.com/openshift-3-demo-part-8-rolling-deployments/

http://docs.spring.io/spring-boot/docs/current/reference/html/howto-database-initialization.html

http://dev.mysql.com/doc/refman/5.7/en/trigger-syntax.html

\url{http://mongodb.github.io/mongo-csharp-driver/2.3/reference/bson/mapping/schema_changes/}

\url{https://www.mongodb.com/blog/post/6-rules-of-thumb-for-mongodb-schema-design-part-1}

\url{https://www.mongodb.com/blog/post/6-rules-of-thumb-for-mongodb-schema-design-part-2}

\url{https://www.mongodb.com/blog/post/6-rules-of-thumb-for-mongodb-schema-design-part-3}