\newcounter{funcReq} %Functional requirement counter
\newcounter{nonFuncReq} %Non functinal requiremet counter

\chapter{Anforderungen}

\section{Aktuelle Anforderungen}

Da die Applikation bereits in Betrieb ist, sind die Anforderungen von den neuen getrennt.

\subsection{Funktionale Anforderungen}

\begin{table}[H]
	\centering
	\caption{Funktionale Anforderungen}
	\begin{tabular}{ | p{2cm} | p{14cm} | }
		\toprule
		{\textbf{ID}} & {\textbf{Beschreibung}} \\
		\midrule
		FA-\arabic{funcReq} \stepcounter{funcReq} & Die Applikation ermöglicht das Bestellen von Paymit für einen Geschäft.  \\ \hline
		FA-\arabic{funcReq} \stepcounter{funcReq} & Es sollen mehrere Varianten auswählbar sein. \\ \hline
		FA-\arabic{funcReq} \stepcounter{funcReq} & Features müssen einfach aktiviert und deaktiviert werden können \\ \hline
		FA-\arabic{funcReq} \stepcounter{funcReq} & Die Applikation soll sich möglichst einfach ausrollen lassen \\
		\bottomrule
	\end{tabular}
\end{table}
	
\subsection{Nicht funktionale Anforderungen}

\begin{table}[H]
	\centering
	\caption{Funktionale Anforderungen}
	\begin{tabular}{ | p{2cm} | p{14cm} | }
		\toprule
		{\textbf{ID}} & {\textbf{Beschreibung}} \\
		\midrule
		NFA-\arabic{nonFuncReq} \stepcounter{nonFuncReq} & Kunde merkt nicht wenn neue Komponenten der Applikation ausgetauscht werden. \\ \hline
		NFA-\arabic{nonFuncReq} \stepcounter{nonFuncReq} & Verfügbarkeit der Anwendung ist 24x7 (SLA). \\ \hline
		NFA-\arabic{nonFuncReq} \stepcounter{nonFuncReq} & Einhaltung der PCI DSS Anforderung bezüglich Umgang mit Daten resp. deren Zugriffsschutz durch Authentifizierung, Authorisierung und entsprechende Netzwerksegmentierung. \\ \hline
		NFA-\arabic{nonFuncReq} \stepcounter{nonFuncReq} & Benutzerfreundliche Oberfläche. \\ \hline
		NFA-\arabic{nonFuncReq} \stepcounter{nonFuncReq} & Die Applikation muss einfach skalierbar sein \\ \hline
		NFA-\arabic{nonFuncReq} \stepcounter{nonFuncReq} & Die Anwendung muss eine hohe Ausfallsicherheit aufweisen \\ \hline
		NFA-\arabic{nonFuncReq} \stepcounter{nonFuncReq} & Bugfixing mittels Vorwärtscommit. Anstelle von Hotfixes wird immer die ganze Applikation neu ausgerollt. \\ \hline
		NFA-\arabic{nonFuncReq} \stepcounter{nonFuncReq} & Die Anwendung lässt sich voll automatisch Ausrollen. \\
		\bottomrule
	\end{tabular}
\end{table}


\section{Neue Anforderungen}

\subsection{Funktionale Anforderungen}

\begin{table}[H]
	\centering
	\caption{Funktionale Anforderungen}
	\begin{tabular}{ | p{2cm} | p{14cm} | }
		\toprule
		{\textbf{ID}} & {\textbf{Beschreibung}} \\
		\midrule
		FA-\arabic{funcReq} \stepcounter{funcReq} & todo \\ \hline
		FA-\arabic{funcReq} \stepcounter{funcReq} & todo \\ \hline
		FA-\arabic{funcReq} \stepcounter{funcReq} & todo \\ \hline
		FA-\arabic{funcReq} \stepcounter{funcReq} & todo \\
		\bottomrule
	\end{tabular}
\end{table}

\subsection{Nicht funktionale Anforderungen}

\begin{table}[H]
	\centering
	\caption{Nicht funktionale Anforderungen}
	\begin{tabular}{ | p{2cm} | p{14cm} | }
		\toprule
		{\textbf{ID}} & {\textbf{Beschreibung}} \\
		\midrule
		NFA-\arabic{nonFuncReq} \stepcounter{nonFuncReq} & Die einzelnen Teile der Applikation lassen sich separat von einander deployment. \\ \hline
		NFA-\arabic{nonFuncReq} \stepcounter{nonFuncReq} & todo \\ \hline
		NFA-\arabic{nonFuncReq} \stepcounter{nonFuncReq} & todo \\ \hline
		NFA-\arabic{nonFuncReq} \stepcounter{nonFuncReq} & todo \\ \hline
		NFA-\arabic{nonFuncReq} \stepcounter{nonFuncReq} & todo \\
		\bottomrule
	\end{tabular}
\end{table}

