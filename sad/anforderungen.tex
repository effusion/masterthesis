\chapter{Anforderungen}

\section{Aktuelle Anforderungen}

Da die Applikation bereits in Betrieb ist, sind die Anforderungen von den neuen getrennt.

\subsection{Funktionale Anforderungen}

\begin{table}[H]
	\centering
	\caption{Funktionale Anforderungen}
	\begin{tabular}{ | p{2cm} | p{14cm} | }
		\toprule
		{\textbf{ID}} & {\textbf{Beschreibung}} \\
		\midrule
		FA-1 & Die Applikation ermöglicht das Bestellen von Paymit für einen Geschäft.  \\ \hline
		FA-2 & Es sollen mehrere Varianten auswählbar sein. \\ \hline
		FA-3 & Features müssen einfach aktiviert und deaktiviert werden können \\ \hline
		FA-4 & Die Applikation soll sich möglichst einfach ausrollen lassen \\
		\bottomrule
	\end{tabular}
\end{table}
	
\subsection{Nicht funktionale Anforderungen}

\begin{table}[H]
	\centering
	\caption{Funktionale Anforderungen}
	\begin{tabular}{ | p{2cm} | p{14cm} | }
		\toprule
		{\textbf{ID}} & {\textbf{Beschreibung}} \\
		\midrule
		FA-1 & Einfach zu bedinender Bestellvorgang. \\ \hline
		FA-2 & Die Applikation muss mehrsprachig sein. \\ \hline
		FA-3 & Teile welche mit Kreditkartendaten arbeiten oder auf solche Dienste zugreifen, müssen in der entsprechenden Netzwerkzone stehen. \\ \hline
		FA-4 & todo \\
		\bottomrule
	\end{tabular}
\end{table}


\section{Neue Anforderungen}

\subsection{Funktionale Anforderungen}

\begin{table}[H]
	\centering
	\caption{Funktionale Anforderungen}
	\begin{tabular}{ | p{2cm} | p{14cm} | }
		\toprule
		{\textbf{ID}} & {\textbf{Beschreibung}} \\
		\midrule
		FA-1 & Die einzelnen Teile der Applikation lassen sich separat von einander deploymen \\ \hline
		FA-2 & todo \\ \hline
		FA-3 & todo \\ \hline
		FA-4 & todo \\
		\bottomrule
	\end{tabular}
\end{table}

\subsection{Nicht funktionale Anforderungen}

\begin{table}[H]
	\centering
	\caption{Nicht funktionale Anforderungen}
	\begin{tabular}{ | p{2cm} | p{14cm} | }
		\toprule
		{\textbf{ID}} & {\textbf{Beschreibung}} \\
		\midrule
		FA-1 & todo \\ \hline
		FA-2 & todo \\ \hline
		FA-3 & todo \\ \hline
		FA-4 & todo \\
		\bottomrule
	\end{tabular}
\end{table}

