\newcounter{nonFuncReq} %Non functinal requiremet counter

\chapter{Anforderungen}

Alle funktionalen Anforderungen sind bereits umgesetzt. Durch die Anpassung der Architektur Richtung Continuous Deployment gibt es aktuell nur nicht funktionale Anforderungen.
	
\begin{table}[H]
	\centering
	\caption{Nicht funktionale Anforderungen}
	\begin{tabular}{ | p{2cm} | p{14cm} | }
		\toprule
		{\textbf{ID}} & {\textbf{Beschreibung}} \\
		\midrule
		NFA-\arabic{nonFuncReq} \stepcounter{nonFuncReq} & Kunde merkt nicht wenn neue Komponenten der Applikation ausgetauscht werden. \\ \hline
		NFA-\arabic{nonFuncReq} \stepcounter{nonFuncReq} & Verfügbarkeit der Anwendung ist 24x7 (SLA). \\ \hline
		NFA-\arabic{nonFuncReq} \stepcounter{nonFuncReq} & Einhaltung der PCI DSS Anforderung bezüglich Umgang mit Daten resp. deren Zugriffsschutz durch Authentifizierung, Authorisierung und entsprechende Netzwerksegmentierung. \\ \hline
		NFA-\arabic{nonFuncReq} \stepcounter{nonFuncReq} & Benutzerfreundliche Oberfläche. \\ \hline
		NFA-\arabic{nonFuncReq} \stepcounter{nonFuncReq} & Die Applikation muss einfach skalierbar sein \\ \hline
		NFA-\arabic{nonFuncReq} \stepcounter{nonFuncReq} & Die Anwendung muss eine hohe Ausfallsicherheit aufweisen \\ \hline
		NFA-\arabic{nonFuncReq} \stepcounter{nonFuncReq} & Bugfixing mittels Vorwärtscommit. Anstelle von Hotfixes wird immer die ganze Applikation neu ausgerollt. \\ \hline
		NFA-\arabic{nonFuncReq} \stepcounter{nonFuncReq} & Die Anwendung lässt sich voll automatisch Ausrollen. \\
		\bottomrule
	\end{tabular}
\end{table}